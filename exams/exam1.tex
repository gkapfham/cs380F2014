%!TEX root=exam1.tex
% mainfile: exam1.tex

\documentclass[12pt]{article}
\textwidth = 6.5in
\textheight = 9.05in
\topmargin 0.0in
\oddsidemargin 0.0in
\evensidemargin 0.0in

% set it so that subsubsections have numbers and they
% are displayed in the TOC (maybe hard to read, might want to disable)

\setcounter{secnumdepth}{3}
\setcounter{tocdepth}{3}

\usepackage{epsfig}

% define widow protection 
        
\def\widow#1{\vskip #1\vbadness10000\penalty-200\vskip-#1}

% define a little section heading that doesn't go with any number

\def\littlesection#1{
\widow{2cm}
\vskip 0.5cm
\noindent{\bf #1}
\vskip 0.1cm
\noindent
}

% A paraphrase mode that makes it easy to see the stuff that shouldn't
% stay in for the final proposal

\newdimen\tmpdim
\long\def\paraphrase#1{{\parskip=0pt\hfil\break
\tmpdim=\hsize\advance\tmpdim by -15pt\noindent%
\hbox to \hsize
{\vrule\hskip 3pt\vrule\hfil\hbox to \tmpdim{\vbox{\hsize=\tmpdim
\def\par{\leavevmode\endgraf}
\obeyspaces \obeylines 
\let\par=\endgraf
\bf #1}}}}}

\renewcommand{\baselinestretch}{1.2}    % must go before the begin of doc

% go with the way that CC sets the margins

\begin{document}

\pagestyle{empty}

% handle widows appropriately
\def\widow#1{\vskip #1\vbadness10000\penalty-200\vskip-#1}

\begin{center}

CS 380: Principles of Database Systems \\
Examination 1 \\
%Thursday, February 22, 2007 \\

\end{center}

\noindent
Answer the five questions in this examination.  You must provide
answers to these questions on a separate sheet of paper.  Please
develop responses that clearly express your ideas in the most succinct
manner possible.  You are not permitted to complete this examination
in conjunction with any of your classmates.  Furthermore, you cannot
consult any outside references during this examination.  If you have
questions concerning these problems, then please visit my office
during the examination period.  If you leave the classroom to take the
exam, then you are responsible for checking the white board for status
updates.

\newpage

\begin{enumerate}

\item ({\bf 10 Points}) Data management is an important issue within
  the discipline of computer science.  Answer the following questions
  about the fundamentals of data management.

\begin{enumerate}

\item ({\bf 2 Points}) One approach to managing data is to simply
  place everything in files and construct a {\em file processing
    system}.  List at least two disadvantages of this scheme.  Your
  response should clearly explain each disadvantage.

%% \item ({\bf 3 Points}) What are three different models for data?  Your
%%   answer should provide a one or two sentence definition of each
%%   model.  You should also create a picture for each of the data
%%   models.

\item ({\bf 4 Points}) What is a {\em two-tiered} architecture?  What
  is a {\em three-tiered} architecture?  Your response should include
  a picture for each of the architectures.

%% \item ({\bf 2 Points}) What is {\em meta-data}?  How does the Java
%%   Database Connectivity (JDBC) interface support the collection of one
%%   or more types of meta-data?

%% \item ({\bf 2 Points}) What is {\em normalization}? What are the
%%   performance trade-off(s) associated with using this technique?

\item ({\bf 4 Points}) Relational databases are often associated with
  both a {\em schema} and an {\em instance}.  Please provide a clear
  definition of these terms, comparing and contrasting them whenever
  it is possible to do so.

\end{enumerate}

\newpage

\item ({\bf 10 Points}) A relational database management system
  (RDBMS) can be used to store data.  A relational database schema
  can contain keys.  Answer the following questions about the 
  RDBMS and the use of keys.

\begin{enumerate}

\item ({\bf 4 Points}) Define the following terms: {\em relation},
  {\em record}, {\em attribute}, and {\em attribute value}.  Your
  definition should include one picture to explain all of these terms.

\item ({\bf 3 Points}) Provide formal definitions of the terms {\em
  super key}, {\em candidate key}, and {\em primary key}.  Whenever
  possible, your response should furnish a diagram or graphical
  example to explain the formal definitions.

%% \item ({\bf 3 Points}) Provide {\bf create table} statements that
%%   define two relations.  You should define these relations so that
%%   $R_i$ is referenced by $R_j$.  This means that $R_j$ must have a
%%   foreign key that references $R_i$.  What should the RDBMS do to
%%   properly handle {\bf insert}s into relation $R_j$?

%% \item ({\bf 2 Points}) Assume that an RDBMS is managing the schema
%%   that you defined in the previous part of this question.  Explain the
%%   different technique(s) that an RDBMS can use to correctly handle the
%%   execution of a {\bf delete} statement in the referenced relation.

%% \item ({\bf 3 Points}) The structured query language (SQL) supports
%%   the definition of {\bf auto increment} attributes.  Please clearly
%%   explain the concept of an attribute that is automatically
%%   incremented.  Your response should include an example of a {\bf
%%     create table} statement that uses the {\bf auto increment}
%%   feature.  Finally, you must furnish three {\bf insert} statements
%%   and then give an example of what the relation will look like after
%%   the execution of these {\bf insert} statements.

\item ({\bf 3 Points}) Most relational database management systems
  separate the abstraction of data into three distinct levels.  After
  giving a name to each of these levels, your response to this
  question should clearly define the level and state whether it is at
  a high, medium, or low level of abstraction.

\end{enumerate}

\newpage

\item ({\bf 10 Points}) The relational theory provides a foundation
  that we can use to understand the structured query language.  Answer
  the following questions about relational theory.

\begin{enumerate}

%% \item ({\bf 3 Point}) Suppose that $R_i$ and $R_j$ are relations.
%%   Under what circumstance(s) is it acceptable to perform the operation
%%   $R_i \cup R_j$?

% \item ({\bf 2 Points}) The relational algebra normally features the
%   $\bowtie$ operator.  What are the input(s) and output(s) of this
%   operator?  How does this operator work?

\item ({\bf 2 Points}) The relational algebra normally features the
  $\times$ operator, called the cartesian product.  What are the
  input(s) and output(s) of this operator?  What is the worst-case time
  complexity of a naive implementation of this operator?

%% After answering the two
%%   previous questions, your response to this problem should furnish an
%%   example of the behavior of the $\bowtie$ operator for specific
%%   input(s).

%% \item ({\bf 3 Points}) Provide formal definitions of the operators
%%   $\sigma$ and $\pi$.  Your response should clearly explain how these
%%   operators are related to the structured query language.

\item ({\bf 3 Points}) What is an {\em aggregate} function?  Your
  response should explain at least two examples of this type of
  function.  Furthermore, your response should furnish a SQL {\bf
    select} statement that uses an aggregate function.

\item ({\bf 2 Points}) Why do {\bf null} values appear in a relational
  database? How should a relational database management system handle
  attributes with {\bf null} values?  

\item ({\bf 3 Points}) Many relational database management systems
  support the definition of an {\em integrity constraint}.  What is an
  integrity constraint?  Your response to this question should furnish
  two examples of this type of constraint.

%% Please provide a complete set of rules that explain how to
%%   incorporate {\bf null} values into the logical operators {\bf and}
%%   and {\bf or}.

\end{enumerate}

\newpage

\item ({\bf 10 Points}) The structured query language (SQL) provides
  facilities for data manipulation and data definition.  Answer the
  following questions about SQL.

\begin{enumerate}

\item ({\bf 2 Point}) Explain the similarities and differences between
  the data types (i) {\tt char(n)} and {\tt varchar(n)} and (ii) {\tt
    int} and {\tt real}.

\item ({\bf 4 Points}) Please provide a SQL {\bf create table} statement
  to define the {\em account} relation.  This relation must have the 
  following attributes:

  \begin{enumerate}

  \item {\em account\_number} as a fixed length char of size 10
    
  \item {\em branch\_name} as a fixed length char of size 15

  \item {\em balance} as a fixed point number with a user-specified
    precision of twelve digits and two decimal places

  \item a primary key that only consists of the {\em account\_number}
    attribute

  \end{enumerate}

\item ({\bf 2 Points}) What is the difference between executing the
  following two structured query language statements?  Your response
  to this question should focus on how these two statements will
  modify the state and structure of a relational database containing
  the relation $R_i$.  Finally, you should classify the statement as
  being a part of either the data definition or the data manipulation
  language.

  \begin{enumerate}

  \item {\bf drop table} $R_i$

  \item {\bf delete from} $R_i$

  \end{enumerate}

\item ({\bf 2 Points}) Relational databases allow for the definition of
  {\em primary} and {\em foreign} keys.  Using an example of one or
  more {\bf create table} statements and a schema diagram, explain the
  terms primary key and foreign key.

%% \item ({\bf 2 Points}) What is a relational {\em view}?  Your response
%%   should explain why views are created.  What are the difficulties
%%   that are often associated with supporting direct modification to a
%%   view?  In particular, your response should clearly explain the
%%   challenges that are associated with handling the {\bf view update}
%%   problem.

\end{enumerate}

\newpage

\item ({\bf 10 Points}) As an alternative to procedural programming
  languages, the structured query language provides many advanced
  features.  Answer the following questions about procedural
  programming and SQL features such as transactions and data types.

\begin{enumerate}

\item ({\bf 2 Points}) The R language for statistical computation
  allows for the implementation of procedural data analysis routines.
  In R, how would you determine which variables exist inside of a data
  frame?  In R, how would you use the {\bf subset} function to extract
  a portion of a data frame?

% \item ({\bf 3 Point}) The structured query language supports the
%   definition of {\bf clob} and {\bf blob} attributes.  What do these
%   data types mean?  When would it be appropriate to use these data
%   types in a relational schema?

\item ({\bf 4 Points}) What is a {\em transaction}?  Why does a
  relational database support transactions?  What are the {\em ACID
    properties} and how are they related to transactions?

\item ({\bf 2 Points}) What are two approaches that an RDBMS could take
  to manage the {\em integrity constraints} during the execution of a
  transaction?  Your answer to this question should clearly explain the
  trade-offs associated with your chosen approaches to constraint
  management during transaction processing.

\item ({\bf 2 Points}) What are the similarities and differences
  between {\em dynamic} SQL and {\em embedded} SQL?  Your response
  should include an example of a language and/or platform that
  provides each type of functionality.

%% \item ({\bf 3 Points}) What are the three {\em canonical concurrency
%%   control} problems?  Your response should select one of these
%%   problems and provide a transaction schedule that clearly
%%   demonstrates how the problem could occur.

\end{enumerate}

\newpage 

%% \item ({\bf +1 Point}) Any student who can answer {\em all} of these
%%   questions correctly will receive one extra bonus point.

%% \begin{enumerate}

%% \item What color is Steve Hazen's favorite hat?

%% \item How much did it cost to replace the battery in Devin Raynor's
%%   green iPod?

%% \item Which student was responsible for yelling out ``You're late!''
%%   when Gavilan Steinman walked into the classroom?

%% \item What knot did I use to tie my tie?

%% \item Which student recently revealed that he {\em never} gives wrong
%%   answers?

%% \end{enumerate}

\end{enumerate}

\end{document}



