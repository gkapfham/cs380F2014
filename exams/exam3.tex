\documentclass[12pt]{article}             
\textwidth = 6.5in
\textheight = 9.05in
\topmargin 0.0in
\oddsidemargin 0.0in
\evensidemargin 0.0in

% set it so that subsubsections have numbers and they
% are displayed in the TOC (maybe hard to read, might want to disable)

\setcounter{secnumdepth}{3}
\setcounter{tocdepth}{3}

\usepackage{epsfig}

% define widow protection 
        
\def\widow#1{\vskip #1\vbadness10000\penalty-200\vskip-#1}

% define a little section heading that doesn't go with any number

\def\littlesection#1{
\widow{2cm}
\vskip 0.5cm
\noindent{\bf #1}
\vskip 0.1cm
\noindent
}

% A paraphrase mode that makes it easy to see the stuff that shouldn't
% stay in for the final proposal

\newdimen\tmpdim
\long\def\paraphrase#1{{\parskip=0pt\hfil\break
\tmpdim=\hsize\advance\tmpdim by -15pt\noindent%
\hbox to \hsize
{\vrule\hskip 3pt\vrule\hfil\hbox to \tmpdim{\vbox{\hsize=\tmpdim
\def\par{\leavevmode\endgraf}
\obeyspaces \obeylines 
\let\par=\endgraf
\bf #1}}}}}

\renewcommand{\baselinestretch}{1.2}    % must go before the begin of doc

% go with the way that CC sets the margins

\vspace*{-1in}

\begin{document}

% handle widows appropriately
\def\widow#1{\vskip #1\vbadness10000\penalty-200\vskip-#1}

\begin{center}

CS 380: Principles of Database Systems \\
Examination 3 \\
%Tuesday, April 17, 2007 \\

\end{center}

\noindent
Answer the following ten examination questions.  You must provide
answers to these questions on a separate sheet of paper.  Please
develop responses that clearly express your ideas in the most succinct
manner possible.  You are not permitted to complete this examination
in conjunction with any of your classmates.  Furthermore, you cannot
consult any outside references during this examination.  If you have
questions concerning the problems that are listed below, please visit
my office during the examination period.  If you leave the classroom
to take the exam, you are responsible for checking the white board for
status updates.

%% , {\bf Score:}
%%   \underline{\hspace*{.2in}} \hspace*{.05in}) 

\begin{enumerate}

\item ({\bf 10 Points}) Data management is an important issue within
  the discipline of computer science.  Answer the following questions
  about the fundamentals of data management.

\begin{enumerate}

\item ({\bf 3 Points}) One approach to managing data is to simply
  place everything in files and construct a {\em file processing
    system}.  List at least three disadvantages of this scheme.  Your
  response should clearly explain each disadvantage.

\item ({\bf 2 Points}) The structured query language (SQL) contains a
  DML and a DDL.  What do these two terms mean?  Your response should
  include an example of both a DML and a DDL statement.

\item ({\bf 2 Points}) It is possible to find data in either a {\em
  declarative} or an {\em imperative} fashion.  Please clearly define
  each of these terms and give an example of both approaches in the
  context of either a Java virtual machine (JVM) heap or an eXtensible
  markup language (XML) document.

%% \item ({\bf 3 Points}) What is {\em meta-data}?  How does the Java
%%   Database Connectivity (JDBC) interface support the collection of
%%   meta-data?  How does a relational database management system (RDBMS)
%%   handle meta-data?  What type of meta-data does an RDBMS collect?

\item ({\bf 3 Points}) What is {\em meta-data}?  How does the Java
  Database Connectivity (JDBC) interface allow a programmer to access
  various types of meta-data? Where does a relational database
  management system (RDBMS) store meta-data?  What type of meta-data
  does an RDBMS collect?

\end{enumerate}

\newpage

\item ({\bf 10 Points}) A relational database management system
  (RDBMS) can be used to store data.  A relational database schema can
  contain keys.  Moreover, there any many different drivers that allow
  standard programs to interact with an RDBMS.  Answer the following
  questions about the RDBMS and the use of keys and drivers.

\begin{enumerate}

\item ({\bf 4 Points}) Define the following terms: {\em record}, {\em
  attribute}, {\em attribute value}, and {\em attribute domain}.
  Using meaningful labels, your definitions should reference a single
  picture that further explains all of these four terms.

\item ({\bf 3 Points}) Provide formal definitions of the terms {\em
  super key}, {\em candidate key}, and {\em primary key}.  Your
  response to this part of the question should include one or more
  graphical examples to fully illustrate the formal definitions.

%% \item ({\bf 3 Points}) Provide {\bf create table} statements that
%%   define two relations.  You should define these relations so that
%%   $R_i$ is referenced by $R_j$.  This means that $R_j$ must have a
%%   foreign key that references $R_i$.  What should the RDBMS do to
%%   properly handle {\bf insert}s into relation $R_j$?

\item ({\bf 3 Points}) The Java database connectivity (JDBC) interface
  furnishes the ability to create and submit a {\em prepared
    statement}.  Furthermore, JDBC also enables a program to execute a
  SQL {\bf select} and receive a {\em result set}.  Using source code
  statements when appropriate, define the terms prepared statement and
  result set.

%% \item ({\bf 2 Points}) The relational algebra includes operations such
%%   as $r \cup s$ and $r - s$.  However, these operations are defined
%%   under the assumption that relations $r$ and $s$ are {\em
%%     compatible}.  What conditions must hold if two relations are
%%   compatible?

\end{enumerate}

\newpage

\item ({\bf 10 Points}) It is important to properly design and
  implement a database-centric application.  Answer the following
  questions about application design and development.

  \begin{enumerate}
    
    \item ({\bf 2 Points}) What are the similarities and differences
      between a three-layer Web architecture and a two-layer
      architecture?  Your response to this question should include a
      picture of each type of architecture.

%%     \item ({\bf 3 Points}) What is a JDBC connection pool?  Your
%%       response should explain why connection pooling is valuable 
%%       and discuss the performance trade-offs associated with the 
%%       use of pools.

%%     \item ({\bf 2 Points}) Suppose that you are writing a Java program
%%       that uses JDBC to connect to an HSQLDB relational database.
%%       Using Java-based psuedo code when appropriate, describe the
%%       steps that occur when you execute the command {\tt java
%%         org.hsqldb.sample.FindFile <name>} (we assume that {\tt
%%         <name>} stands for the signature of a file in your home
%%       account).

    \item ({\bf 3 Points}) What is a database trigger?  Please
      describe a single circumstance in which it would be valuable to
      use triggers.  Finally, you should identify a situation in which
      it would {\em not} be appropriate to use triggers.  As part of
      your answer, you must carefully explain {\em why} it would or
      would not be wise to use triggers.  Whenever possible, your
      response should include diagrams and/or concrete examples.

    %% \item ({\bf 3 Points}) The structured query language provides both
    %%   the {\tt array} and the {\tt multiset} data types.  What are the
    %%   similarities and differences between these two types?  Your
    %%   response should contain an example of a variable that could be
    %%   properly represented by each of these data types.

    \item ({\bf 3 Points}) What is a {\em transaction}?  Why does a
      relational database support transactions?  What approach should
      an RDBMS take to manage the {\em integrity constraints} during
      the execution of a transaction?  Your answer to this question
      should clearly explain the trade-offs associated with your
      chosen approach to constraint management during transaction
      processing.

    \item ({\bf 2 Points}) The transaction manager for a relational
      database management system normally aims to enforce the {\em
        ACID properties}.  What are these properties?

    \end{enumerate}
      
\newpage

\item ({\bf 10 Points}) The structured query language (SQL) is
  regularly used in data management situations.  Answer the following
  questions about SQL.

  \begin{enumerate}
    
    \item ({\bf 2 Points}) Suppose that you used the {\bf create
      table} statement to create a relation $rel_j$ that contains
      attributes $A$, $B$, and $C$.  What SQL statement would you
      execute if you wanted to add another attribute $D$ to this
      relation?  What SQL statement would you run if you needed to
      remove $rel_j$'s attribute $A$?

    \item ({\bf 8 Points)} Propose your own (simple) relation by
      providing a {\bf create table} statement to construct this
      relation.  Next, you should provide a sample instance of your
      schema by populating it with data and giving a diagram (you do
      not have to provide the {\bf insert} statements that would
      perform the population).

      Then, you must write an SQL {\bf update}, {\bf insert}, and {\bf
        delete} statement and explain what each of these statements
      are supposed to do.  These commands should be accompanied by the
      before and after state of the relation.  Finally, please provide
      a SQL {\bf select} query that uses one or more aggregate
      functions and a {\bf where} clause.  You must produce the output
      of the {\bf select} statement.

    \end{enumerate}

\newpage

\item ({\bf 10 Points}) There are several techniques that may be used
  to implement database applications.  Answer the following questions
  that are related to application development and data management in
  the context of relational databases.

  \begin{enumerate}

    %% \item ({\bf 5 Points}) What is the difference between a {\em
    %%   static} Web site and a {\em dynamic} Web site?  How do databases
    %%   support the implementation of one (or both) of these types of
    %%   sites?  As a Web developer, what are the trade-offs that you
    %%   would evaluate when deciding whether to implement your next Web
    %%   site in a static or dynamic fashion?  Your response should
    %%   include an example of both types of Web site.

  \item ({\bf 2 Points}) What is the difference between executing the
    following two structured query language statements?  Your response
    to this question should focus on how these two statements will
    modify the state and structure of a relational database containing
    the relation $R_i$.  Finally, you should classify the statement as
    being a part of either the data definition or the data
    manipulation language.

    \begin{enumerate}

    \item {\bf drop table} $R_i$

    \item {\bf delete from} $R_i$

    \end{enumerate}

  \item ({\bf 3 Points}) What is a relational {\em view}?  Your
    response should explain why views are created.  What are the
    difficulties that are often associated with supporting direct
    modification to a view?  In particular, your response should
    clearly explain the challenges that are associated with handling
    the {\em view update} problem.
    
%%     \item ({\bf 2 Points}) What is the difference between a {\em
%%       static} Web site and a {\em dynamic} Web site?  How do databases
%%       support the implementation of one (or both) of these types of
%%       sites?

    %% \item ({\bf 3 Points)} What is a {\em stored procedure}?  Where is
    %%   a stored procedure located?  What are the benefits associated
    %%   with using a stored procedure?

%%     \item ({\bf 5 Points}) Suppose that you are performing load
%%       testing for a database-driven Web site.  How will {\em
%%         throughput} and {\em response time} change as you increase the
%%       number of clients that are concurrently using the Web site?
%%       What can you do to improve the performance of a database-driven
%%       Web site?

    \item ({\bf 5 Points}) Suppose that you are performing load
      testing for a database-driven Web site.  How will {\em
        throughput} and {\em response time} change as you increase the
      number of clients that are concurrently using the Web site?
      What can you do to improve the performance of a database-driven
      Web site?  Your answer to this question should include a graph
      that plots the relationship between throughput and response time
      as the number of concurrent clients increases.

    \end{enumerate}

\newpage

\item ({\bf 10 Points}) You must consider security and performance
  when you are using a data management system.  Answer the following
  questions about these important areas.

  \begin{enumerate}
    
    \item ({\bf 2 Points}) What is an SQL injection attack? What steps
      can a programmer take to prevent from taking place and/or
      mitigate the risk of this type of attack?

    \item ({\bf 3 Points}) What is {\em authentication}?  {\em
      Authorization}?  How does the structured query language support
      these two types of security features?

    %% \item ({\bf 4 Points}) What are the three {\em canonical
    %%   concurrency control} problems?  Your response should select one
    %%   of these problems and provide a transaction schedule that
    %%   clearly demonstrates how the problem could occur.  What are the
    %%   performance trade-offs associated with using transactions and a
    %%   {\em transaction manager} in a database-centric application?

    \item ({\bf 2 Points}) Many tables in a relational database have a
      primary key.  When you create a relational database, you have to
      consider the efficiency of the data type for the primary key.
      For both of the following examples, you should explain which
      data type would support more efficient queries.  You must
      carefully justify your response to each part of this question.

      \begin{enumerate}

      \item {\tt varchar(10)} versus {\tt varchar(100)}

      \item {\tt int} versus {\tt real}

      \end{enumerate}

    \item ({\bf 3 Points}) Ehcache is a widely used distributed cache
      for general purpose caching in the Java programming language.
      Please clearly explain how you could use Ehcache to improve the
      performance of a Web-based database application.  Your response
      to this question should include one or more architecture
      diagrams that clearly show the location of the instance of the
      Ehcache.

    \end{enumerate}

\newpage

%% \item ({\bf 10 Points}) Answer the following questions about how
%%   database applications process large data sets and use transactions
%%   to manage concurrent executions.

%% \begin{enumerate}

%% \item ({\bf 7 Points}) We can define the {\em good time} of a
%%   distributed data processing system using Equation~(\ref{goodput}).
%%   Furthermore, we can define the {\em allocation efficiency} of a
%%   distributed data processor's handling of a specific workload using
%%   Equation~(\ref{eff}).

%%     \begin{equation}
%%       time_{good} = time_{compute} - time_{checkpoint} - 
%%       time_{roll back} \label{goodput}
%%     \end{equation} 

%%     \begin{equation}
%%       Allocation \; \mbox{{\em efficiency}} = \frac{time_{good}}{time_{{compute}}} 
%%       \label{eff}
%%     \end{equation}
    
%%     In this question, you may assume that the placement time for a
%%     distributed data processor is negligible and processor pre-emption
%%     never occurs.  After placement, a processor can always execute for
%%     $60$ minutes of total computation time.  Since it is possible for
%%     a data processor to crash, we assume that checkpoints will save
%%     the state of the computation.  Each time the data processor
%%     checkpoints, it requires $2$ minutes of checkpoint time.  %\newline

%%     If a data processor crashes and rollback is required, then this
%%     will take a total of $20$ minutes.  If there is one checkpoint,
%%     then a rollback occurs with a $45\%$ probability.  When there are
%%     two checkpoints, the chance of a rollback is $30\%$.  Finally,
%%     when a maximum of three checkpoints occur, the chance of rollback
%%     is $25\%$. %\newline

%%     Given the previous information, which checkpointing frequency is
%%     best able to maximize allocation efficiency for the distributed
%%     data processor?  In order to receive full credit, you must show
%%     all of the work that enabled you to arrive at your final answer.
%%     No prior knowledge of distributed data processing is required in
%%     order to correctly answer this question.

%%   \item ({\bf 3 Points}) A transaction manager that uses locking might
%%     enter into deadlock.  A {\em wait for graph} (WFG) represents the
%%     dependencies between the transactions in a database system so that
%%     $T_i \rightarrow T_j$ means that transaction $T_i$ ``waits for''
%%     $T_j$.  Is the database system in Figure~\ref{figure:wfg} in
%%     deadlock?  If your response is {\em yes}, then please state how
%%     you would resolve the deadlock.  If you response is {\em no}, then
%%     indicate what edge(s) would need to be added in order to cause the
%%     system to go into deadlock.  Your response should present a
%%     general algorithm for detecting whether or not a transaction
%%     processor system, represented by a WFG, is or is not in deadlock.

%%   \end{enumerate}
 
\newpage

\item ({\bf 10 Points}) The eXtensible Markup Language (XML) can be
  used to store semi-structured data.  Answer the following questions
  about XML.

  \begin{enumerate}
    
    \item ({\bf 4 Points}) XQuery is often described as being the
      ``standard'' query language for XML.  What are the FLWOR
      expressions that you can write in XQuery?

      \item ({\bf 3 Points}) What is DOM?  What is SAX?  How are these
      two technologies similar to and different from each other?  Your
      response should contain a detailed analysis of the trade-offs 
      associated with using these different APIs.

    \item ({\bf 3 Points}) Recent empirical results suggest that the
      XML format is very {\em compressible}.  What does it mean if a
      data storage format is highly compressible?  Why is XML very
      compressible?  You should support your response to this question
      with concrete examples and equations.  In particular, you should
      furnish an equation that shows how to calculate the percent
      change in the size of an uncompressed and compressed XML file.

    \end{enumerate}

\newpage

\item ({\bf 10 Points}) The eXtensible Markup Language (XML) is a
  popular data interchange format.  Answer the following questions
  about XML.

  \begin{enumerate}
    
    \item ({\bf 4 Points}) Figure~\ref{figure:xml} furnishes an
      example of an XML file.  Using parts of this document in your
      response, please define the following terms.

      \begin{enumerate}

        \item data

        \item meta-data

        \item tag

        \item attribute

      \end{enumerate}

  \item ({\bf 3 Points}) Many authors suggest that XML is a very {\em
    verbose} data storage format.  Do you agree with this statement?
    Why or why not? Using the XML file in Figure~\ref{figure:xml}, you
    should provide evidence to support your response.

  \item ({\bf 3 Points}) In some situations, it may be useful to store
    an XML document inside of a database.  If you were required to use
    a relational database, then how would you store an XML document
    inside of it?  If you could pick any type of database, then how
    would you store the XML file?

  \end{enumerate}

%% \item ({\bf 10 Points}) The eXtensible Markup Language (XML) can be
%%   used to store semi-structured data.  Answer the following questions
%%   about XML.

%%   \begin{enumerate}
    
%%     \item ({\bf 3 Points}) XQuery is often described as being the
%%       ``standard'' query language for XML.  What are the FLWOR
%%       expressions that you can write in XQuery?

%%     \item ({\bf 2 Points}) Many authors suggest that XML is a very
%%       {\em verbose} data storage format.  Do you agree with this
%%       statement?  Why or why not?  You should provide evidence to
%%       support your response.

%%     \item ({\bf 3 Points}) What is DOM?  What is SAX?  How are these
%%       two technologies similar to and different from each other?  Your
%%       response should contain a detailed analysis of the trade-offs 
%%       associated with using these different APIs.

%%     \item ({\bf 2 Points}) Recent empirical results suggest that the
%%       XML format is very {\em compressible}.  What does it mean if a
%%       data storage format is highly compressible?  Why is XML very
%%       compressible?

%%     \end{enumerate}

\newpage

\item ({\bf 10 Points}) Answer the following questions about database
  applications and their use during data mining and warehousing.

  \begin{enumerate}

  \item ({\bf 2 Points}) Your text book generally classifies database
    applications into two distinct categories.  What are these two
    categories?  Your response to this part of the question should
    include an example of each type of application.

  %% \item ({\bf 4 Points}) A data warehouse normally uses either a {\em
  %%   push} or a {\em pull} method for gathering data.  Please define
  %%   these two strategies for populating the many databases that are
  %%   normally a part of the warehouse.  Your response to this question
  %%   should include a detailed diagram of the architecture of a data
  %%   warehouse.

  \item ({\bf 4 Points}) Many data warehousing applications implement
    something called {\em OLAP}.  After defining each of the letters
    in OLAP, please explain how it connects to data warehousing.

  \item ({\bf 4 Points}) Data warehouses often contain large data sets
    with a mixture of {\em categorical} and {\em numerical}
    attributes.  As part of your response to this question, please
    define and give an example of these two types of attributes.

  %% \item ({\bf 2 Points}) The creators of MonetDB claim that a
  %%   column-oriented database supports effective data compression.
  %%   What is a column-oriented storage scheme and why does it enable
  %%   data compression in a warehouse?

\end{enumerate}

\newpage

\item ({\bf 10 Points}) Answer the following questions about the use
  of machine learning algorithms during the process of data mining and
  data warehousing.

  \begin{enumerate}

  \item ({\bf 4 Points}) Machine learning algorithms generally focus
    on {\em prediction} tasks in either a {\em classification} or a
    {\em regression} framework.  Please define these two terms and
    give an example of two separate data mining problems that would
    respectively involve a regression or a classification.

  \item ({\bf 4 Points}) Figure~\ref{figure:tree} gives an example of
    a regression tree that was constructed by the {\em recursive
      partitioning} algorithm (the leaves of this tree correspond to
    values of the ``perf'' attribute).  As part of your response to
    this question, you should explain how to interpret a tree of this
    nature and state two or three interesting trends that are exposed
    by this tree model.  An inspection of the tree reveals that it
    does not use all of the attributes -- can you suggest why this may
    be the case?

  \item ({\bf 2 Points}) How would you use the regression tree in
    Figure~\ref{figure:tree} to perform prediction?  Your response to
    this question should furnish a concrete example that explains how
    to predict with the tree that describes CPU performance.

\end{enumerate}

\end{enumerate}

\begin{figure}[p]

\vspace*{-.5in}

\centering
%\includegraphics{wfg}

{\footnotesize 
\begin{verbatim}
<Publications xmlns="http://www.pitt.edu/APS">
  <Publication>
    <DocUid>2011-002</DocUid>
    <Title>CONFLuEnCE: CONtinuous workFLow ExeCution Engine</Title>
    <Authors>
        <Author>Panayiotis Neophytou</Author>
        <Author>Panos K. Chrysanthis</Author>
        <Author>Alexandros Labrinidis</Author>
    </Authors>
    <DatePublished>2011-06</DatePublished>
    <PublishedIn>SIGMOD 2011</PublishedIn>
    <PubType>Demonstration</PubType>
    <Projects>
        <Project>AQSIOS</Project>
    </Projects>
    <SubjectAreas>
        <SubjectArea>Data Streams</SubjectArea>
        <SubjectArea>Data Integration</SubjectArea>
        <SubjectArea>Workflow Management</SubjectArea>
    </SubjectAreas>
    <Attachments>
        <Attachment>
            <FileName>mod668-neophytou.pdf</FileName>
            <Size>211655</Size>
            <Url>http://rodos.cs.pitt.edu:8090/aps/online/download.do?uid=02b15e8229acc195</Url>
        </Attachment>
        <Attachment>
            <FileName>mod668-neophytou.pdf.zip</FileName>
            <Size>160203</Size>
            <Url>http://rodos.cs.pitt.edu:8090/aps/online/download.do?uid=02b15e8228406ac0</Url>
        </Attachment>
    </Attachments>
  </Publication>
 </Publications>
\end{verbatim} }

\caption{An Example of an eXtensible Markup Language (XML) Document.}
\label{figure:xml}
\end{figure}

\begin{figure}[p]
\centering

\vspace*{-.6in}

\begin{center}
\includegraphics[totalheight=0.6\textheight]{cputree}
\end{center}

\vspace*{-.65in}

\caption{A Regression Tree that Models a CPU Performance Data Set.}
\label{figure:tree}
\vspace*{-.25in}
\end{figure}

\begin{figure}[p]

  \centering

  \begin{tabular}{c | c}

    {\bf Variable} & {\bf Meaning} \\ \hline

    syct & cycle time in nanoseconds \\ \hline

    mmin & minimum main memory in kilobytes \\ \hline

    mmax & maximum main memory in kilobytes \\ \hline

    cach & cache size in kilobytes \\ \hline

    chmin & minimum number of channels \\ \hline

    chmax & maximum number of channels \\ \hline

    perf & performance of 209 different CPUs

  \end{tabular}

\end{figure}

\end{document}



