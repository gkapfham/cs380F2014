%!TEX root=cs380F2014-lab3.tex
% mainfile: cs380F2014-lab3.tex

%!TEX root=cs380F2014-lab1.tex
% mainfile: cs380F2014-lab1.tex 

% Typical usage (all UPPERCASE items are optional):
%       \input 580pre
%       \begin{document}
%       \MYTITLE{Title of document, e.g., Lab 1\\Due ...}
%       \MYHEADERS{short title}{other running head, e.g., due date}
%       \PURPOSE{Description of purpose}
%       \SUMMARY{Very short overview of assignment}
%       \DETAILS{Detailed description}
%         \SUBHEAD{if needed} ...
%         \SUBHEAD{if needed} ...
%          ...
%       \HANDIN{What to hand in and how}
%       \begin{checklist}
%       \item ...
%       \end{checklist}
% There is no need to include a "\documentstyle."
% However, there should be an "\end{document}."
%
%===========================================================
\documentclass[11pt,twoside,titlepage]{article}
%%NEED TO ADD epsf!!
\usepackage{threeparttop}
\usepackage{graphicx}
\usepackage{latexsym}
\usepackage{color}
\usepackage{listings}
\usepackage{fancyvrb}
%\usepackage{pgf,pgfarrows,pgfnodes,pgfautomata,pgfheaps,pgfshade}
\usepackage{tikz}
\usepackage[normalem]{ulem}
\tikzset{
    %Define standard arrow tip
%    >=stealth',
    %Define style for boxes
    oval/.style={
           rectangle,
           rounded corners,
           draw=black, very thick,
           text width=6.5em,
           minimum height=2em,
           text centered},
    % Define arrow style
    arr/.style={
           ->,
           thick,
           shorten <=2pt,
           shorten >=2pt,}
}
\usepackage[noend]{algorithmic}
\usepackage[noend]{algorithm}
\newcommand{\bfor}{{\bf for\ }}
\newcommand{\bthen}{{\bf then\ }}
\newcommand{\bwhile}{{\bf while\ }}
\newcommand{\btrue}{{\bf true\ }}
\newcommand{\bfalse}{{\bf false\ }}
\newcommand{\bto}{{\bf to\ }}
\newcommand{\bdo}{{\bf do\ }}
\newcommand{\bif}{{\bf if\ }}
\newcommand{\belse}{{\bf else\ }}
\newcommand{\band}{{\bf and\ }}
\newcommand{\breturn}{{\bf return\ }}
\newcommand{\mod}{{\rm mod}}
\renewcommand{\algorithmiccomment}[1]{$\rhd$ #1}
\newenvironment{checklist}{\par\noindent\hspace{-.25in}{\bf Checklist:}\renewcommand{\labelitemi}{$\Box$}%
\begin{itemize}}{\end{itemize}}
\pagestyle{threepartheadings}
\usepackage{url}
\usepackage{wrapfig}
% removing the standard hyperref to avoid the horrible boxes
%\usepackage{hyperref}
\usepackage[hidelinks]{hyperref}
% added in the dtklogos for the bibtex formatting
\usepackage{dtklogos}
%=========================
% One-inch margins everywhere
%=========================
\setlength{\topmargin}{0in}
\setlength{\textheight}{8.5in}
\setlength{\oddsidemargin}{0in}
\setlength{\evensidemargin}{0in}
\setlength{\textwidth}{6.5in}
%===============================
%===============================
% Macro for document title:
%===============================
\newcommand{\MYTITLE}[1]%
   {\begin{center}
     \begin{center}
     \bf
     CMPSC 380\\Principles of Database Systems\\
     Fall 2014
     \medskip
     \end{center}
     \bf
     #1
     \end{center}
}
%================================
% Macro for headings:
%================================
\newcommand{\MYHEADERS}[2]%
   {\lhead{#1}
    \rhead{#2}
    %\immediate\write16{}
    %\immediate\write16{DATE OF HANDOUT?}
    %\read16 to \dateofhandout
    \def \dateofhandout {September 3, 2014}
    \lfoot{\sc Handed out on \dateofhandout}
    %\immediate\write16{}
    %\immediate\write16{HANDOUT NUMBER?}
    %\read16 to\handoutnum
    \def \handoutnum {3}
    \rfoot{Handout \handoutnum}
   }

%================================
% Macro for bold italic:
%================================
\newcommand{\bit}[1]{{\textit{\textbf{#1}}}}

%=========================
% Non-zero paragraph skips.
%=========================
\setlength{\parskip}{1ex}

%=========================
% Create various environments:
%=========================
\newcommand{\PURPOSE}{\par\noindent\hspace{-.25in}{\bf Purpose:\ }}
\newcommand{\SUMMARY}{\par\noindent\hspace{-.25in}{\bf Summary:\ }}
\newcommand{\DETAILS}{\par\noindent\hspace{-.25in}{\bf Details:\ }}
\newcommand{\HANDIN}{\par\noindent\hspace{-.25in}{\bf Hand in:\ }}
\newcommand{\SUBHEAD}[1]{\bigskip\par\noindent\hspace{-.1in}{\sc #1}\\}
%\newenvironment{CHECKLIST}{\begin{itemize}}{\end{itemize}}


\usepackage[compact]{titlesec}

\begin{document}
\MYTITLE{Laboratory Assignment Three: Procedural Programming and File Processing Systems}
\MYHEADERS{Laboratory Assignment Three}{Due: September 24, 2014}

\section*{Introduction}

In this laboratory assignment, we will use the procedural (or, imperative) approach to implementing a simple file
processing system.  In this assignment you will familiarize yourself with the steps that a scientist would take to
analyze, manipulate, and visualize a data set. In particular, you will learn how to use the R language for statistical
computation to manage and visualize a file-based data set. You will also develop a preliminary understanding of how how
to write simple procedures that select a subset of data from a larger data set. Also, you will try to summarize data
sets using functions such as the mean and median. Once you have a better understanding the challenges associated with
the use of imperative programming techniques during data analysis and management, you will also investigate the steps
needed to produce visualizations of a file-based data set.

\section*{Installing and Configuring the Data Sets}

During the completion of this laboratory assignment, we will rely upon some of the tools and data sets provided by Luis
Torgo's book entitled ``Data Mining with R: Learning with Case Studies''. A copy of the book will be available for your
consultation throughout the laboratory session and then held on reserve in my outer office. You can also learn more
about this book by visiting the following Web site: \url{http://www.dcc.fc.up.pt/~ltorgo/DataMiningWithR/}.

As previously mentioned, this assignment invites you to implement procedural (or, imperative) methods---in the R
language for statistical computation---to answer questions about a real-world data set.  To start this laboratory
assignment, you should start using the command-line-based and interactive environment of the R language. You can do this
by typing ``{\tt R}'' into your terminal window. After making sure that you already have a Web browser open, you should
next type the command ``{\tt help.start()}'' in order to load the help system for R. You can use this Web-based help
environment throughout the laboratory session as you learn new commands for imperative data management. In addition, you
can learn more about R by visiting Web sites such as \url{http://www.statmethods.net/} and
\url{http://www.cyclismo.org/tutorial/R/}.

To access the data files needed for this assignment, you must install the software package for the companion textbook.
Since this process may takes a long time, you should type the following command in your R shell right away: ``{\tt
  install.packages("DMwR")}''. Once you have pressed enter, you will see that many lines scroll in your terminal window
as R performs a network install of all the required packages, including the needed data files.  When this process
finishes, you can load the library by typing ``{\tt library(DMwR)}''. You should also repeat these steps for the
``{\tt Hmisc}'' package.

In order to effectively use R to analyze the data in a file, you need to learn several basic commands. You can view the
manual for a specific command, say {\tt subset}, by typing ``{\tt ?subset}'' at the R prompt. In this phase of the
assignment, you should learn how to use several R commands and write a short description of their input, output, and
behavior, including one concrete example of the command in action with the data set discussed at a later stage of the
assignment. Please see the course instructor if you have a question about how to use a data manipulation command in R.
In particular, you must study the following commands available in the R language:

\begin{itemize}
    \itemsep.1in
  \item {\tt attach}
  \item {\tt names}
  \item {\tt head}
  \item {\tt mean}
  \item {\tt median}
  \item {\tt subset}
  \item {\tt ls}
  \item {\tt summarize} or {\tt summary}
\end{itemize}

  % \item Now, you need to test to see if you can authenticate with the Bitbucket servers.  First, show the course
  %   instructor that you have correctly configured your Bitbucket account.  Now, ask the instructor to share the course's
  %   Git repository with you.  Open a terminal window on your workstation and change into the directory where you will
  %   store your files for this course.  For instance, you might make a {\tt cs380F2014/} directory that will contain the
  %   Git repository that I will always use to share files with you.  Once you have done so, please type the following
  %   command: {\tt }.  If everything worked correctly, you
  %   should be able to download all of the files that you will need to use for this laboratory assignment. Please resolve
  %   any problems that you encountered by first reviewing the Bitbucket documentation and then discussing the matter
  %   with your team. If you are still not able to run the {\tt git clone} command, then please see the instructor.

  % \item Using your terminal window, you should browse the files that are in this Git repository.  In particular, please
  %   look in the {\tt labs/lab1/src/} directory and use Vim to study the two Java programs that you find.  Remember, the
  %   {\tt cd} command allows you to change into a directory. 

\section*{Procedural Exploration of a Data Set}

After you have learned more about how each one of the aforementioned commands works, you should learn more about the
{\tt algae} data set that is part of the {\tt DMwR} library. What are the names of the attributes in this data set? How
many attributes are in the data set? How many rows are in the data set? While you must use the R programming language to
answer these questions, you can check your responses by visiting the ``Algae Data Set Description'' in the UCI Machine
Learning Repository and reading the ``Predicting Algae Blooms'' chapter in Torgo's book. 

As you learn more about this data set, you will notice that the {\tt a1} attribute gives the frequency number of a
harmful algae known as ``a1'' in the data set. Please note that the data set does not contain any information about the
name or the characteristics of this type of alga. Each value in the {\tt a1} attribute corresponds to a ``frequency''
that this algae was found in the specified environment; in this case, small values are better since they indicate the
presence of less of this harmful algae. Using the mean and the median values of the frequencies for alga {\tt a1}, is
this algae more likely to bloom in rivers classified as ``small'', ``medium'', or ``large''?

The {\tt a2} and {\tt a3} attributes respectively give the frequency number of a harmful algae known as ``a2'' and
``a3'' in the data set. Using the mean and the median values of the frequencies for algae {\tt a2} and {\tt a3}, are
these algae more likely to bloom in ``small'', ``medium'', or ``large'' rivers?

Additionally, the {\tt algae} data set also contains details about the speed of the rivers, as stored in the {\tt speed}
attribute. Using the mean and the median values for the ``a1'', ``a2'', and ``a3'' algae, are these algae more
likely to bloom in rivers with a ``low'', ``medium'', or ``high'' speed?

The {\tt Cl} attribute in the {\tt algae} data set describes the mean amount of chlorophyll in the rivers. What is the
relationship between the mean value of chlorophyll and the amount of ``a1'', ``a2'', and ``a3'' in the rivers?
For instance, if a river has a ``high'' amount of chlorophyll, does this mean that it will contain a ``high'' or a
``low'' amount of each algae? Why do you think this is the case?

There are many other interesting trends evident in the algae data set. Using any combination of R commands, please
identify and try to explain one additional trend. As you are completing this last part of the laboratory assignment, it
may be helpful to visualize the {\tt algae} data set so that you can easily and quickly spot new and interesting
phenomena. To create a wide variety of different graphs, you can explore the use of the ``Lattice'' package that is
available after you type the ``{\tt library(lattice)}'' command. You can learn more about Lattice by visiting the
package's Web site that is available at \url{http://lmdvr.r-forge.r-project.org/}. Once you have learned how to use
Lattice to produce simple data visualizations in R, you should construct at least one graph to accompany your response
to one of the previously stated questions.

% At this point, you should go into the
% {\tt cs112s2014-share} repository and use the {\tt cp -r} command to copy the entire {\tt labs/} directory from the {\tt
% cs112s2014-share} repository to {\tt cs380F2014-<your user name>}.  Once the files are in your own Git repository,
% please use the {\tt git add} and {\tt git commit} commands to add them correctly. If you do not know how to use the {\tt
% git add} and {\tt git commit} commands in the terminal window, please learn more about them by searching on the
% Internet, talking about them with your team, and discussing them with the course instructor.

\section*{Summary of the Required Deliverables}

  This assignment invites you to submit one printed version of the following deliverables:

  \vspace*{-.05in}
  \begin{enumerate}
    \setlength{\itemsep}{0pt}
    \item A screenshot showing a small selection of the output seen when installing ``{\tt DMwR}''.
    \item The description of the input, output, and behavior for the required R commands.
    \item Answers to all of the questions posed, with supporting output and evidence as appropriate. 
    \item At least one supporting visualization of some trend that you found in the data set. 
    \item A commentary on the challenges that you faced and the way(s) that you overcame them.
  \end{enumerate}

\vspace*{-.1in} 

In adherence to the Honor Code, students should complete this assignment on an individual basis. While it is appropriate
for students in this class to have high-level conversations about the assignment, it is necessary to distinguish
carefully between the student who discusses the principles underlying a problem with others and the student who produces
assignments that are identical to, or merely variations on, someone else's work.  Deliverables that are nearly identical
to the work of others will be taken as evidence of violating Allegheny College's \mbox{Honor Code}.

% Before you turn in this assignment, you also must ensure that the course instructor has read access to your Bitbucket
% repository that is named according to the convention {\tt cs380F2014-<your user name>}.  Please see the instructor if
% you have any questions about this assignment. 




\end{document}
