%!TEX root=cs380F2014-lab3.tex
% mainfile: cs380F2014-lab3.tex

\input{labspre.tex}

\usepackage[compact]{titlesec}

\begin{document}
\MYTITLE{Laboratory Assignment Three \\ Combining Imperative and
  Declarative Programming: \\ Implementation and Empirical Evaluation}
\MYHEADERS{Laboratory Assignment Three}{Due: September 24, 2014}

\section*{Introduction}

In this laboratory assignment, we will learn how to use a variant of the structured query language to search for
matching objects in the heap of a Java virtual machine, with a particular focus on the ``Java Objects for SQL'' (JoSQL)
library. That is, this assignment introduces a method for combining the imperative and declarative approaches to
managing data. After you have finished configuring your development environment to run a Java program that uses JoSQL,
you will implement your own benchmarking framework that will support experiments that measure the performance of
different JoSQL features.  Finally, you will perform a comprehensive empirical study, analyze and visualize the results,
write up the results in a formal report, and give a team presentation describing your benchmarks, the experiment
design, and the empirical results.

\section*{Accessing the JoSQL Library}

Since you will complete this assignment in a team, you should first gather together with your team members, introducing
yourselves to each other and explaining your interests and areas of expertise. Next, one member of your team should
create a Git repository for this laboratory assignment, sharing it with all of the team members and the course
instructor. Please make sure that every member of your team has write access to this Git repository hosted by Bitbucket.
Additionally, each student should ask the course instructor to give you read access to the {\tt cs380f2014-share}
repository and then use the {\tt git clone git@bitbucket.org:gkapfham/cs380f2014-share.git} command to obtain access to
the resources for this laboratory assignment.

\section*{Combining Imperative and Declarative Programming with JoSQL}

\begin{sloppypar}

  Start exploring the {\tt cs380f2014-share} repository, specifically studying the files in the {\tt
    cs380F2014-share/labs/lab3/} directory. What files are available? What is the purpose of these files? Make sure that
  you used a text editor to review the source code of the {\tt FileFinder.java} program. Once you understand how this
  program works, you should change the specified directory to one that is contained in your home account and consider
  modifying the file matching pattern some something instead of {\tt LIKE `\%ja\%'}. Next, you should set your {\tt
    CLASSPATH} environment variable to include both the current working directory and the two Java archives in the {\tt lib/}
  directory. Before you move on to the next phase of this assignment, please make sure that every member of your team is
  able to compile and run the {\tt FileFinder.java} program. In addition, every member of your team should be able to
  explain why this program works the way that it does.

\end{sloppypar}

\section*{Experimental Performance Evaluation of Declarative Querying}

The existence of JoSQL makes it possible to frame and answer many interesting questions about the performance associated
with using a variant of the structured query language in an object-oriented program.  For instance, would the
performance of {\tt FileFinder} vary if, to store Java objects, you used a {\tt LinkedList} or a {\tt Vector} instead of
an {\tt ArrayList}?  JoSQL includes many different types of SQL statements---what is the performance of different
querying constructs?  Moreover, it would also be interesting to compare the performance of JoSQL to a hand-coded
alternative that uses iteration constructs to directly search a data structure for matching objects.

As the final part of this laboratory assignment, you are responsible for implementing a simple benchmarking framework
for JoSQL.  Your framework should contain, at minimum, five benchmarks that can be run in different configurations.  You
should design your benchmarks so that they will enable you to answer specific questions about the performance trade-offs
associated with using the structured query language for Java objects.  After completing the implementation of your
benchmarks, you should run them to collect performance results concerning time and, if possible, space overheads.
Please execute the benchmarks multiple times and record the results in a structured fashion; each team should
consider storing their data in comma-separated value (CSV) files.

In order to analyze the results, you should construct data tables and visualizations.  For instance, if your team stored
their data in CSV files, you can easily write a program in the R language for statistical computing that can read in the
these files and subsequently analyze and visualize it. After developing an understanding of the fundamental
trends in your data sets, your team should write a report that details your findings.  The report should explain your
benchmarks, the evaluation metrics, and the key empirical results.  In particular, you should observe the performance
trends and explain why these trends are evident in your data set.  Whenever possible, you should comment on the
practical implications of your results for software developers who may use JoSQL.  

You are responsible for working in a team to coordinate the experiments that you conduct.  In particular, you should
make sure that each member of the team implements different benchmarks and then conducts experiments in a uniform
manner.  After the members of the team have individually completed the experiments and recorded and analyzed their
results, you should meet together to discuss the trends that you found in the data sets.  While students are responsible
for writing their own reports that focus on their own data sets and analyses, they should work together in a team to
prepare a single presentation for the laboratory session on which this assignment is due.  After explaining the purpose
and configuration of your experimental study, the presentation should highlight the similarities and differences in the
results developed by each team member.  Finally, you should draw some conclusions about which algorithmic approach
and/or data management technique is the fastest.  Please note that each team member is responsible for independently
conducting their own experiments, writing their own report, and adequately participating in the completion of the slides
and the final presentation.  </p>


\section*{Summary of the Required Deliverables}

  This assignment invites you to submit one printed version of the following deliverables:

  \vspace*{-.05in}
  \begin{enumerate}
    \setlength{\itemsep}{0pt}
    \item A screenshot showing a small selection of the output seen when installing ``{\tt DMwR}''.
    \item The description of the input, output, and behavior for the required R commands.
    \item Answers to all of the questions posed, with supporting output and evidence as appropriate. 
    \item At least one supporting visualization of some trend that you found in the data set. 
    \item A commentary on the challenges that you faced and the way(s) that you overcame them.
  \end{enumerate}

\vspace*{-.1in} 

In adherence to the Honor Code, students should complete this assignment on an individual basis. While it is appropriate
for students in this class to have high-level conversations about the assignment, it is necessary to distinguish
carefully between the student who discusses the principles underlying a problem with others and the student who produces
assignments that are identical to, or merely variations on, someone else's work.  Deliverables that are nearly identical
to the work of others will be taken as evidence of violating Allegheny College's \mbox{Honor Code}.

% Before you turn in this assignment, you also must ensure that the course instructor has read access to your Bitbucket
% repository that is named according to the convention {\tt cs380F2014-<your user name>}.  Please see the instructor if
% you have any questions about this assignment. 




\end{document}
