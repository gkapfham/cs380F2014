%!TEX root=cs380F2014-lab3.tex
% mainfile: cs380F2014-lab3.tex

%!TEX root=cs380F2014-lab1.tex
% mainfile: cs380F2014-lab1.tex 

% Typical usage (all UPPERCASE items are optional):
%       \input 580pre
%       \begin{document}
%       \MYTITLE{Title of document, e.g., Lab 1\\Due ...}
%       \MYHEADERS{short title}{other running head, e.g., due date}
%       \PURPOSE{Description of purpose}
%       \SUMMARY{Very short overview of assignment}
%       \DETAILS{Detailed description}
%         \SUBHEAD{if needed} ...
%         \SUBHEAD{if needed} ...
%          ...
%       \HANDIN{What to hand in and how}
%       \begin{checklist}
%       \item ...
%       \end{checklist}
% There is no need to include a "\documentstyle."
% However, there should be an "\end{document}."
%
%===========================================================
\documentclass[11pt,twoside,titlepage]{article}
%%NEED TO ADD epsf!!
\usepackage{threeparttop}
\usepackage{graphicx}
\usepackage{latexsym}
\usepackage{color}
\usepackage{listings}
\usepackage{fancyvrb}
%\usepackage{pgf,pgfarrows,pgfnodes,pgfautomata,pgfheaps,pgfshade}
\usepackage{tikz}
\usepackage[normalem]{ulem}
\tikzset{
    %Define standard arrow tip
%    >=stealth',
    %Define style for boxes
    oval/.style={
           rectangle,
           rounded corners,
           draw=black, very thick,
           text width=6.5em,
           minimum height=2em,
           text centered},
    % Define arrow style
    arr/.style={
           ->,
           thick,
           shorten <=2pt,
           shorten >=2pt,}
}
\usepackage[noend]{algorithmic}
\usepackage[noend]{algorithm}
\newcommand{\bfor}{{\bf for\ }}
\newcommand{\bthen}{{\bf then\ }}
\newcommand{\bwhile}{{\bf while\ }}
\newcommand{\btrue}{{\bf true\ }}
\newcommand{\bfalse}{{\bf false\ }}
\newcommand{\bto}{{\bf to\ }}
\newcommand{\bdo}{{\bf do\ }}
\newcommand{\bif}{{\bf if\ }}
\newcommand{\belse}{{\bf else\ }}
\newcommand{\band}{{\bf and\ }}
\newcommand{\breturn}{{\bf return\ }}
\newcommand{\mod}{{\rm mod}}
\renewcommand{\algorithmiccomment}[1]{$\rhd$ #1}
\newenvironment{checklist}{\par\noindent\hspace{-.25in}{\bf Checklist:}\renewcommand{\labelitemi}{$\Box$}%
\begin{itemize}}{\end{itemize}}
\pagestyle{threepartheadings}
\usepackage{url}
\usepackage{wrapfig}
% removing the standard hyperref to avoid the horrible boxes
%\usepackage{hyperref}
\usepackage[hidelinks]{hyperref}
% added in the dtklogos for the bibtex formatting
\usepackage{dtklogos}
%=========================
% One-inch margins everywhere
%=========================
\setlength{\topmargin}{0in}
\setlength{\textheight}{8.5in}
\setlength{\oddsidemargin}{0in}
\setlength{\evensidemargin}{0in}
\setlength{\textwidth}{6.5in}
%===============================
%===============================
% Macro for document title:
%===============================
\newcommand{\MYTITLE}[1]%
   {\begin{center}
     \begin{center}
     \bf
     CMPSC 380\\Principles of Database Systems\\
     Fall 2014
     \medskip
     \end{center}
     \bf
     #1
     \end{center}
}
%================================
% Macro for headings:
%================================
\newcommand{\MYHEADERS}[2]%
   {\lhead{#1}
    \rhead{#2}
    %\immediate\write16{}
    %\immediate\write16{DATE OF HANDOUT?}
    %\read16 to \dateofhandout
    \def \dateofhandout {September 3, 2014}
    \lfoot{\sc Handed out on \dateofhandout}
    %\immediate\write16{}
    %\immediate\write16{HANDOUT NUMBER?}
    %\read16 to\handoutnum
    \def \handoutnum {3}
    \rfoot{Handout \handoutnum}
   }

%================================
% Macro for bold italic:
%================================
\newcommand{\bit}[1]{{\textit{\textbf{#1}}}}

%=========================
% Non-zero paragraph skips.
%=========================
\setlength{\parskip}{1ex}

%=========================
% Create various environments:
%=========================
\newcommand{\PURPOSE}{\par\noindent\hspace{-.25in}{\bf Purpose:\ }}
\newcommand{\SUMMARY}{\par\noindent\hspace{-.25in}{\bf Summary:\ }}
\newcommand{\DETAILS}{\par\noindent\hspace{-.25in}{\bf Details:\ }}
\newcommand{\HANDIN}{\par\noindent\hspace{-.25in}{\bf Hand in:\ }}
\newcommand{\SUBHEAD}[1]{\bigskip\par\noindent\hspace{-.1in}{\sc #1}\\}
%\newenvironment{CHECKLIST}{\begin{itemize}}{\end{itemize}}


\usepackage[compact]{titlesec}

\begin{document}
\MYTITLE{Laboratory Assignment Three \\ Combining Imperative and
  Declarative Programming: \\ Implementation and Empirical Evaluation}
\MYHEADERS{Laboratory Assignment Three}{Due: September 24, 2014}

\section*{Introduction}

In this laboratory assignment, we will learn how to use a variant of the structured query language to search for
matching objects in the heap of a Java virtual machine, with a particular focus on the ``Java Objects for SQL'' (JoSQL)
library. That is, this assignment introduces a method for combining the imperative and declarative approaches to
managing data. After you have finished configuring your development environment to run a Java program that uses JoSQL,
you will implement your own benchmarking framework that will support experiments that measure the performance of
different JoSQL features.  Finally, you will perform a comprehensive empirical study, analyze and visualize the results,
write up the results in a formal report, and give a group presentation describing your benchmarks, the experiment
design, and the empirical results.

\section*{Accessing the JoSQL Library}

Since you will complete this assignment in a team, you should first gather together with your team members, introducing
yourselves to each other and explaining your interests and areas of expertise. Next, one member of your team should
create a Git repository for this laboratory assignment, sharing it with all of the team members and the course
instructor. Please make sure that every member of your team has write access to the Git repository hosted by Bitbucket.
Additionally, each student should ask the course instructor to share the {\tt cs380f2014-share} repository with you and
then use the {\tt git clone git@bitbucket.org:gkapfham/cs380f2014-share.git} command to obtain access to the resources
for this laboratory assignment.

\section*{Summary of the Required Deliverables}

  This assignment invites you to submit one printed version of the following deliverables:

  \vspace*{-.05in}
  \begin{enumerate}
    \setlength{\itemsep}{0pt}
    \item A screenshot showing a small selection of the output seen when installing ``{\tt DMwR}''.
    \item The description of the input, output, and behavior for the required R commands.
    \item Answers to all of the questions posed, with supporting output and evidence as appropriate. 
    \item At least one supporting visualization of some trend that you found in the data set. 
    \item A commentary on the challenges that you faced and the way(s) that you overcame them.
  \end{enumerate}

\vspace*{-.1in} 

In adherence to the Honor Code, students should complete this assignment on an individual basis. While it is appropriate
for students in this class to have high-level conversations about the assignment, it is necessary to distinguish
carefully between the student who discusses the principles underlying a problem with others and the student who produces
assignments that are identical to, or merely variations on, someone else's work.  Deliverables that are nearly identical
to the work of others will be taken as evidence of violating Allegheny College's \mbox{Honor Code}.

% Before you turn in this assignment, you also must ensure that the course instructor has read access to your Bitbucket
% repository that is named according to the convention {\tt cs380F2014-<your user name>}.  Please see the instructor if
% you have any questions about this assignment. 




\end{document}
