%!TEX root=cs380F2014-lab2.tex
% mainfile: cs380F2014-lab2.tex

\input{labspre.tex}

\usepackage[compact]{titlesec}

\begin{document}
\MYTITLE{Laboratory Assignment Two: Procedural Programming and File Processing Systems}
\MYHEADERS{Laboratory Assignment Two}{Due: September 10, 2014}

\section*{Introduction}

In this laboratory assignment, we will use the procedural (or, imperative) approach to implementing a simple file
processing system.  In this assignment you will familiarize yourself with the steps that a scientist would take to
analyze, manipulate, and visualize a data set. In particular, you will learn how to use the R language for statistical
computation to manage and visualize a file-based data set. You will also develop a preliminary understanding of how how
to write simple procedures that select a subset of data from a larger data set. Also, you will try to summarize data
sets using functions such as the mean and median. Once you have a better understanding the challenges associated with
the use of imperative programming techniques during data analysis and management, you will also investigate the steps
needed to produce visualizations of a file-based data set.

\section*{Configuring Git and Bitbucket}

During this laboratory assignment and subsequent assignments, we will securely communicate with the Bitbucket.org
servers that will host all of our projects.  In this laboratory assignment, we will perform all of the steps to
configure the accounts on the departmental servers and the Bitbucket service.  Throughout the assignment, you should
refer to the following Web site for additional information:
\url{https://confluence.atlassian.com/display/BITBUCKET/Bitbucket+101}.  As you will be required to turn in a report
describing each step in this assignment, please be sure to keep a record of all of the tasks that you complete and the
challenges that you face.

\begin{enumerate}
	
  \item If you have never done so before, you must use the {\tt ssh-keygen} program to create secure-shell keys that
    support your communication with the Bitbucket servers.  Type {\tt man ssh-keygen} and talk, at a high level, with
    the members of the class to learn more about how to use this program.  What files does {\tt ssh-keygen} produce?
    Where does this program store these files?

  \item If you do not already have a Bitbucket account, please go to the Bitbucket Web site and create one --- 
    make sure that you use your {\tt allegheny.edu} email address so that you can create an unlimited number of free
    Bitbucket repositories. Then, upload your ssh key to Bitbucket.

    \item At this point, you can learn more about Git both by discussing it, at a high level, with your colleagues and
      consulting Web sites like \url{http://try.github.io/} and \url{http://gitimmersion.com/}.  You should
      ensure that you fully understand how to use the following Git commands:

      \begin{enumerate} 
        \item {\tt git init}
        \item {\tt git status}
        \item {\tt git add} 
        \item {\tt git commit}
        \item {\tt git push}
        \item {\tt git pull} 
      \end{enumerate}

  % \item Now, you need to test to see if you can authenticate with the Bitbucket servers.  First, show the course
  %   instructor that you have correctly configured your Bitbucket account.  Now, ask the instructor to share the course's
  %   Git repository with you.  Open a terminal window on your workstation and change into the directory where you will
  %   store your files for this course.  For instance, you might make a {\tt cs380F2014/} directory that will contain the
  %   Git repository that I will always use to share files with you.  Once you have done so, please type the following
  %   command: {\tt }.  If everything worked correctly, you
  %   should be able to download all of the files that you will need to use for this laboratory assignment. Please resolve
  %   any problems that you encountered by first reviewing the Bitbucket documentation and then discussing the matter
  %   with your team. If you are still not able to run the {\tt git clone} command, then please see the instructor.

  % \item Using your terminal window, you should browse the files that are in this Git repository.  In particular, please
  %   look in the {\tt labs/lab1/src/} directory and use Vim to study the two Java programs that you find.  Remember, the
  %   {\tt cd} command allows you to change into a directory. 

\end{enumerate}

\section*{Creating and Populating a New Git Repository}

Now that you have learned more about using Git, you should make a new repository in the {\tt cs380F2014/} directory that
you previously created.  First, make a new directory called {\tt cs380F2014\-<your user name>}. Now, you can use the
{\tt git init} command to turn this directory into a local Git repository.  After completing this step, you should
make a {\tt lab1/} directory and then create a plain-text file (e.g., a {\tt .txt}, {\tt .md}, or {\tt .tex} file) with
a name containing the word ``features''.  For instance, if you choose to write this report in \LaTeX, then you would
name your file {\tt features.tex}. Finally, you must create an additional plain-text file with a name including the word
``tools''.  Again, if you decide to write this report in \LaTeX, then you could create a file called {\tt
  tools.tex}. 

Next, you should use the Bitbucket Web site to create a repository that has the same name as the local directory and
local repository (i.e., {\tt cs380F2014-<your user name>}).  You must follow Bitbucket's instructions to push the code
and tags in your local repository to the remote one. When you are finished with this step, you should see in your Web
browser that the Bitbucket servers are storing the two text files. Once the Git repository contains the correct files,
you should share your Bitbucket repository with the course instructor, whose Bitbucket user name is ``gkapfham''.

% At this point, you should go into the
% {\tt cs112s2014-share} repository and use the {\tt cp -r} command to copy the entire {\tt labs/} directory from the {\tt
% cs112s2014-share} repository to {\tt cs380F2014-<your user name>}.  Once the files are in your own Git repository,
% please use the {\tt git add} and {\tt git commit} commands to add them correctly. If you do not know how to use the {\tt
% git add} and {\tt git commit} commands in the terminal window, please learn more about them by searching on the
% Internet, talking about them with your team, and discussing them with the course instructor.

\section*{Investigating Data Management Tools}

Using your ``features'' file in the Git repository, please write a short one-page document that explains the features
that you think data management tools should provide.  In advance of listing and explaining these features, your
document should clearly define the term ``data management tool''. Whenever possible, you should rank the features in the
order of their importance.

In the ``tools'' file, you should prepare a comprehensive listing of ten free and open-source tools that provide data
management facilities.  This report should give the name of the tool, the Web site(s) and/or papers that you referenced
to learn more about it, and a detailed description of the features that it provides.  As you explain each tool, you
should comment on whether or not it furnishes any of the features that you mentioned in your ``features'' document.
Whenever possible, the report should comment on the type of data that the tool aims to manage. 

When writing these two documents, you must adhere to the Honor Code statement as articulated in the syllabus.  In
particular, you must take care to ensure that you properly cite your sources and that you use your own words to explain
both the desired features and the ten tools. 

\section*{Summary of the Required Deliverables}

  This assignment invites you to submit one printed version of the following deliverables:

  \vspace*{-.05in}
  \begin{enumerate}
    \setlength{\itemsep}{0pt}
    \item A description of the steps that a user must take to configure Git and Bitbucket.
    \item A description of the inputs, outputs, and behavior of the six aforementioned Git commands.
    \item A complete description of the features that data management tools should provide.
    \item A comprehensive listing of ten free and open-source data management tools.
  \end{enumerate}

\vspace*{-.1in} 

Before you turn in this assignment, you also must ensure that the course instructor has read access to your Bitbucket
repository that is named according to the convention {\tt cs380F2014-<your user name>}.  Please see the instructor if
you have any questions about this assignment. 




\end{document}
