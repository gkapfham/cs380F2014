%!TEX root=cs380F2014-fp.tex 
% mainfile: cs380F2014-fp.tex 

%!TEX root=cs380F2014-lab1.tex
% mainfile: cs380F2014-lab1.tex 

% Typical usage (all UPPERCASE items are optional):
%       \input 580pre
%       \begin{document}
%       \MYTITLE{Title of document, e.g., Lab 1\\Due ...}
%       \MYHEADERS{short title}{other running head, e.g., due date}
%       \PURPOSE{Description of purpose}
%       \SUMMARY{Very short overview of assignment}
%       \DETAILS{Detailed description}
%         \SUBHEAD{if needed} ...
%         \SUBHEAD{if needed} ...
%          ...
%       \HANDIN{What to hand in and how}
%       \begin{checklist}
%       \item ...
%       \end{checklist}
% There is no need to include a "\documentstyle."
% However, there should be an "\end{document}."
%
%===========================================================
\documentclass[11pt,twoside,titlepage]{article}
%%NEED TO ADD epsf!!
\usepackage{threeparttop}
\usepackage{graphicx}
\usepackage{latexsym}
\usepackage{color}
\usepackage{listings}
\usepackage{fancyvrb}
%\usepackage{pgf,pgfarrows,pgfnodes,pgfautomata,pgfheaps,pgfshade}
\usepackage{tikz}
\usepackage[normalem]{ulem}
\tikzset{
    %Define standard arrow tip
%    >=stealth',
    %Define style for boxes
    oval/.style={
           rectangle,
           rounded corners,
           draw=black, very thick,
           text width=6.5em,
           minimum height=2em,
           text centered},
    % Define arrow style
    arr/.style={
           ->,
           thick,
           shorten <=2pt,
           shorten >=2pt,}
}
\usepackage[noend]{algorithmic}
\usepackage[noend]{algorithm}
\newcommand{\bfor}{{\bf for\ }}
\newcommand{\bthen}{{\bf then\ }}
\newcommand{\bwhile}{{\bf while\ }}
\newcommand{\btrue}{{\bf true\ }}
\newcommand{\bfalse}{{\bf false\ }}
\newcommand{\bto}{{\bf to\ }}
\newcommand{\bdo}{{\bf do\ }}
\newcommand{\bif}{{\bf if\ }}
\newcommand{\belse}{{\bf else\ }}
\newcommand{\band}{{\bf and\ }}
\newcommand{\breturn}{{\bf return\ }}
\newcommand{\mod}{{\rm mod}}
\renewcommand{\algorithmiccomment}[1]{$\rhd$ #1}
\newenvironment{checklist}{\par\noindent\hspace{-.25in}{\bf Checklist:}\renewcommand{\labelitemi}{$\Box$}%
\begin{itemize}}{\end{itemize}}
\pagestyle{threepartheadings}
\usepackage{url}
\usepackage{wrapfig}
% removing the standard hyperref to avoid the horrible boxes
%\usepackage{hyperref}
\usepackage[hidelinks]{hyperref}
% added in the dtklogos for the bibtex formatting
\usepackage{dtklogos}
%=========================
% One-inch margins everywhere
%=========================
\setlength{\topmargin}{0in}
\setlength{\textheight}{8.5in}
\setlength{\oddsidemargin}{0in}
\setlength{\evensidemargin}{0in}
\setlength{\textwidth}{6.5in}
%===============================
%===============================
% Macro for document title:
%===============================
\newcommand{\MYTITLE}[1]%
   {\begin{center}
     \begin{center}
     \bf
     CMPSC 380\\Principles of Database Systems\\
     Fall 2014
     \medskip
     \end{center}
     \bf
     #1
     \end{center}
}
%================================
% Macro for headings:
%================================
\newcommand{\MYHEADERS}[2]%
   {\lhead{#1}
    \rhead{#2}
    %\immediate\write16{}
    %\immediate\write16{DATE OF HANDOUT?}
    %\read16 to \dateofhandout
    \def \dateofhandout {September 3, 2014}
    \lfoot{\sc Handed out on \dateofhandout}
    %\immediate\write16{}
    %\immediate\write16{HANDOUT NUMBER?}
    %\read16 to\handoutnum
    \def \handoutnum {3}
    \rfoot{Handout \handoutnum}
   }

%================================
% Macro for bold italic:
%================================
\newcommand{\bit}[1]{{\textit{\textbf{#1}}}}

%=========================
% Non-zero paragraph skips.
%=========================
\setlength{\parskip}{1ex}

%=========================
% Create various environments:
%=========================
\newcommand{\PURPOSE}{\par\noindent\hspace{-.25in}{\bf Purpose:\ }}
\newcommand{\SUMMARY}{\par\noindent\hspace{-.25in}{\bf Summary:\ }}
\newcommand{\DETAILS}{\par\noindent\hspace{-.25in}{\bf Details:\ }}
\newcommand{\HANDIN}{\par\noindent\hspace{-.25in}{\bf Hand in:\ }}
\newcommand{\SUBHEAD}[1]{\bigskip\par\noindent\hspace{-.1in}{\sc #1}\\}
%\newenvironment{CHECKLIST}{\begin{itemize}}{\end{itemize}}


\usepackage[compact]{titlesec}

\begin{document} \MYTITLE{Final Project: Advanced Topics in Data Management} 
\MYHEADERS{Final Project}{Due: Friday, December 12, 2014 at 5:00 pm}

% \vspace*{-.32in}

\section*{Introduction}

Throughout the semester, you have learned more about the basics of data management by studying, in a hands-on fashion,
topics such as the use of the structured query language, the implementation of database applications, and the creation
and parsing of XML files.  This final project invites you to explore, in greater detail, an advanced topic in the field
of databases. You will learn more about how to implement, evaluate, and/or simulate key components or aspects of a
database. 

Your project should result in a detailed report that is, ideally, formatted with the \LaTeX\ text processing language
and suitable for publication in a conference or workshop.  The report should include a description of why the chosen
topic is important and discuss the implementation and experimentation that you undertook.  The written material should
be precise, formal, appropriately formatted, grammatically correct, informative, and interesting.  The source code that
you write must be carefully documented and tested.  If you install and use a data management framework, the steps for
installation and use should be clearly documented. 

\section*{Description of the Topics}

Each member of the class is invited to pick one of the following projects.  Please note that an individual selecting the
student-designed project must first discuss their idea with the instructor, during today's laboratory session, and
receive feedback and then final approval.  Please note that you are responsible for ensuring the feasibility of the
project that you propose.

\begin{enumerate}

  \item {\bf Evaluating the Performance of Relational Databases}: Use and/or extend systems like PolePosition,
    Database-Benchmark, or HammerDB to measure the performance of different relational databases.  This project would
    consider both in-memory databases like HSQLDB and standard databases like SQLite.  Students who select this project
    may also implement their own performance benchmarks for relational databases.  The benchmarks and
    the empirical study should, if possible, consider both the time and space overhead of the database.








    
\end{enumerate} 



\section*{Final Project Deadlines}

This assignment invites you to submit printed and signed versions of the following deliverables: 

\vspace*{-.05in}
\begin{enumerate}

  \itemsep0in

  \item {\bf Project Assigned:} April 7, 2014

    After meeting with your team members, pick a topic for your final project.  Remember, if you select the
    student-designed project, you must first have your project verified by the course instructor.  Next, make sure that
    you create a Git repository that can be accessed by all members of the team. Finally, write and submit a
    one-paragraph description of your idea.

  \item {\bf Project Proposal}: April 14, 2014

    You and your team will submit a one-page proposal that describes the idea for your final project.  The proposal
    should have an informative title, an abstract, a description of the main idea, a plan for completing the work, and
    an initial listing of the roles for \mbox{each member}. 

  \item {\bf Status Update}: April 21, 2014

    You and your team will submit a one-page status update that describes the finalized roles of each team member,
    explains the tasks that you have already completed, and outlines a plan for ensuring that all members can easily
    contribute to the successful finish of the final project.

  \item {\bf Project Presentations}: April 28, 2014

    Using slides and demonstrations as needed, you and your team will give a short ten-minute presentation of your final
    results and participate in a five minute question and answer session.

\end{enumerate}
\vspace*{-.05in}

In adherence to the honor code, students should complete this assignment while only collaborating with those students
who are a part of their chosen team. While it is appropriate for students in this class to have high-level
conversations about the assignment with non-team members, it is necessary to distinguish carefully between the team that
discusses the principles underlying a problem with others and the team that produces an assignment that is identical to,
or merely variations on, the work of someone not on the team.  As such, deliverables that are nearly identical to the
work of others will be taken as evidence of violating the \mbox{Honor Code}.  

  \end{document}
