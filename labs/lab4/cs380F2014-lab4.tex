%!TEX root=cs380F2014-lab4.tex
% mainfile: cs380F2014-lab4.tex

%!TEX root=cs380F2014-lab1.tex
% mainfile: cs380F2014-lab1.tex 

% Typical usage (all UPPERCASE items are optional):
%       \input 580pre
%       \begin{document}
%       \MYTITLE{Title of document, e.g., Lab 1\\Due ...}
%       \MYHEADERS{short title}{other running head, e.g., due date}
%       \PURPOSE{Description of purpose}
%       \SUMMARY{Very short overview of assignment}
%       \DETAILS{Detailed description}
%         \SUBHEAD{if needed} ...
%         \SUBHEAD{if needed} ...
%          ...
%       \HANDIN{What to hand in and how}
%       \begin{checklist}
%       \item ...
%       \end{checklist}
% There is no need to include a "\documentstyle."
% However, there should be an "\end{document}."
%
%===========================================================
\documentclass[11pt,twoside,titlepage]{article}
%%NEED TO ADD epsf!!
\usepackage{threeparttop}
\usepackage{graphicx}
\usepackage{latexsym}
\usepackage{color}
\usepackage{listings}
\usepackage{fancyvrb}
%\usepackage{pgf,pgfarrows,pgfnodes,pgfautomata,pgfheaps,pgfshade}
\usepackage{tikz}
\usepackage[normalem]{ulem}
\tikzset{
    %Define standard arrow tip
%    >=stealth',
    %Define style for boxes
    oval/.style={
           rectangle,
           rounded corners,
           draw=black, very thick,
           text width=6.5em,
           minimum height=2em,
           text centered},
    % Define arrow style
    arr/.style={
           ->,
           thick,
           shorten <=2pt,
           shorten >=2pt,}
}
\usepackage[noend]{algorithmic}
\usepackage[noend]{algorithm}
\newcommand{\bfor}{{\bf for\ }}
\newcommand{\bthen}{{\bf then\ }}
\newcommand{\bwhile}{{\bf while\ }}
\newcommand{\btrue}{{\bf true\ }}
\newcommand{\bfalse}{{\bf false\ }}
\newcommand{\bto}{{\bf to\ }}
\newcommand{\bdo}{{\bf do\ }}
\newcommand{\bif}{{\bf if\ }}
\newcommand{\belse}{{\bf else\ }}
\newcommand{\band}{{\bf and\ }}
\newcommand{\breturn}{{\bf return\ }}
\newcommand{\mod}{{\rm mod}}
\renewcommand{\algorithmiccomment}[1]{$\rhd$ #1}
\newenvironment{checklist}{\par\noindent\hspace{-.25in}{\bf Checklist:}\renewcommand{\labelitemi}{$\Box$}%
\begin{itemize}}{\end{itemize}}
\pagestyle{threepartheadings}
\usepackage{url}
\usepackage{wrapfig}
% removing the standard hyperref to avoid the horrible boxes
%\usepackage{hyperref}
\usepackage[hidelinks]{hyperref}
% added in the dtklogos for the bibtex formatting
\usepackage{dtklogos}
%=========================
% One-inch margins everywhere
%=========================
\setlength{\topmargin}{0in}
\setlength{\textheight}{8.5in}
\setlength{\oddsidemargin}{0in}
\setlength{\evensidemargin}{0in}
\setlength{\textwidth}{6.5in}
%===============================
%===============================
% Macro for document title:
%===============================
\newcommand{\MYTITLE}[1]%
   {\begin{center}
     \begin{center}
     \bf
     CMPSC 380\\Principles of Database Systems\\
     Fall 2014
     \medskip
     \end{center}
     \bf
     #1
     \end{center}
}
%================================
% Macro for headings:
%================================
\newcommand{\MYHEADERS}[2]%
   {\lhead{#1}
    \rhead{#2}
    %\immediate\write16{}
    %\immediate\write16{DATE OF HANDOUT?}
    %\read16 to \dateofhandout
    \def \dateofhandout {September 3, 2014}
    \lfoot{\sc Handed out on \dateofhandout}
    %\immediate\write16{}
    %\immediate\write16{HANDOUT NUMBER?}
    %\read16 to\handoutnum
    \def \handoutnum {3}
    \rfoot{Handout \handoutnum}
   }

%================================
% Macro for bold italic:
%================================
\newcommand{\bit}[1]{{\textit{\textbf{#1}}}}

%=========================
% Non-zero paragraph skips.
%=========================
\setlength{\parskip}{1ex}

%=========================
% Create various environments:
%=========================
\newcommand{\PURPOSE}{\par\noindent\hspace{-.25in}{\bf Purpose:\ }}
\newcommand{\SUMMARY}{\par\noindent\hspace{-.25in}{\bf Summary:\ }}
\newcommand{\DETAILS}{\par\noindent\hspace{-.25in}{\bf Details:\ }}
\newcommand{\HANDIN}{\par\noindent\hspace{-.25in}{\bf Hand in:\ }}
\newcommand{\SUBHEAD}[1]{\bigskip\par\noindent\hspace{-.1in}{\sc #1}\\}
%\newenvironment{CHECKLIST}{\begin{itemize}}{\end{itemize}}


\usepackage[compact]{titlesec}

\begin{document}
\MYTITLE{Laboratory Assignment Four \\ Visualizing Relational Database
  Schemas}
\MYHEADERS{Laboratory Assignment Four}{Due: October 16, 2014}

\section*{Introduction}

When either designing a new database or attempting to understand an existing one, it is often useful to produce a
visualization of the relational schema.  A technical diagram of a schema helps a database designer to see the
relationships that already exist in a complex database, thus aiding understanding and the identification of defects.

\section*{Accessing the JoSQL Library}


\section*{Combining Imperative and Declarative Programming with JoSQL}


\section*{Experimental Performance Evaluation of Declarative Querying}
\vspace*{-.2in}



\section*{Summary of the Required Deliverables}

You and your team should always use your Git repository, hosted by Bitbucket, to store the source code of your
benchmarking framework, all of the CSV data files, the final report, and the presentation slides. The repository must be
shared with the course instructor and the version control log should accurately reflect each student's contribution to
this assignment. In addition, this assignment invites your team to submit one printed version of the following
deliverables; each member should write and submit their own version of the first deliverable. 

% \vspace*{-.05in}

\begin{enumerate}
    % \setlength{\itemsep}{0pt}
  \item A five paragraph commentary on the work that each team member completed. 
  \item A full-featured description of the key features provided by the JoSQL library.
  \item The output from each team member's use of the {\tt FileFinder.java} program.
  \item The properly formatted and documented version of the source code for your benchmarks.
  \item The CSV data files resulting from all of the runs of your benchmarking framework.
  \item A comprehensive report describing your benchmarks, the evaluation metrics, and the results.
    \item The presentation slides to support your team's five minute presentation.
\end{enumerate}

% \vspace*{-.1in} 

In adherence to the Honor Code, students should complete this assignment while exclusively collaborating with the
members of their team. While it is appropriate for students in this class---who are not in the same team---to have
high-level conversations about the assignment, it is necessary to distinguish carefully between the team that discusses
the principles underlying a problem with another team and the team that produces an assignment that is identical to, or
merely a variation on, the work of another team.  Deliverables from one team that are nearly identical to the work of
another team will be taken as evidence of violating Allegheny College's \mbox{Honor Code}.

% Before you turn in this assignment, you also must ensure that the course instructor has read access to your Bitbucket
% repository that is named according to the convention {\tt cs380F2014-<your user name>}.  Please see the instructor if
% you have any questions about this assignment. 




\end{document}
