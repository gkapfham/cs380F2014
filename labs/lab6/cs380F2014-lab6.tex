%!TEX root=cs380F2014-lab6.tex
% mainfile: cs380F2014-lab6.tex

\input{labspre.tex}

\usepackage[compact]{titlesec}

\begin{document}
\MYTITLE{Laboratory Assignment Six \\ Implementing Database-Centric
Applications}
\MYHEADERS{Laboratory Assignment Six}{Due: October 29, 2014}

\section*{Introduction}

Chapter 5 of your textbook introduces advanced concepts in the structure query language (SQL), with a focus on topics
like accessing a database from a programming language. In this laboratory assignment, you will leverage the example in
Figure 5.1 of the textbook to ultimately implement your own database-centric application based on a schema that you
visualized and manipulated in the past two assignments. In addition to using a simple Java program that uses a
databases, you will explore more about the configuration and use of the HyperSQL database engine.

\vspace*{-.05in}
\section*{Learning About the HyperSQL Database Engine}

Please visit \url{http://hsqldb.org/} to learn more about the HyperSQL database engine, often called HSQLDB. This
database management system (DBMS) claims to be a ``100\% Java Database''---what are the benefits and drawbacks associated with
having a database that is completely implemented in the Java programming language? What are the features and
limitations associated with using this DBMS? When a database application is not running and interacting with the
database, how does HSQLDB persist the data values? What applications currently use HSQLDB?

\vspace*{-.05in}
\section*{Compiling and Using the FindFile Program}

Please use a terminal window to navigate to the ``share'' Git repository that we use for this course and type the ``{\tt
  git pull}'' command. Once the download is finished, please study the source code that is available in the {\tt
  FindFile.java} program.  You and your partner should discuss this program, drawing diagrams and referencing the
textbook as necessary, until you understand its structure and behavior. It is important to note that {\tt FindFile} must
run in two separate phases: an initialize phase and then a searching phase. Why does the program work in this fashion?
Please pick a suitable directory in your file system (i.e., one that contains many files with different names), and run
both phases of {\tt FindFile}. What are the inputs and outputs of this program?

If you want to compile and run the {\tt FindFile} program, you need to set your {\tt CLASSPATH} environment variable to
contain the {\tt hsqldb.jar} file that is in the {\tt lib/} directory. It is worth noting that {\tt FindFile.java} will
compile without {\tt hsqldb.jar} being available from the {\tt CLASSPATH}---and yet, the program will not run without
this environment variable being set correctly.  Why is this the case? Once you understand how this program interacts
with the HyperSQL database engine, you should run it multiple times to learn how it persists data on the file system.
What does the schema of {\tt FindFile}'s database look like? Can you visualize it with {\tt sqlt-graph}? You and your
partner should see the instructor if you are not able to get the {\tt FindFile} to work correctly.

\vspace*{-.05in}
\section*{Implementing a Database-Centric Application}

After you and your partner completely understand how {\tt FindFile} works, you should pick a relational database schema
that one of you used in one of the past two assignments. Leveraging the schema visualizations and data manipulation
language (DML) statements that you previously created, you and your partner should think of a full-featured program that
might interact with the database. In adherence to the ``dynamic SQL'' approach to implementing database-centric
applications and in recognition of the strengths and weaknesses of HSQLDB, you should design either a command-line or
graphical interface for your program. What operations will your application support? How will these methods use DML
statements like {\tt select}, {\tt update}, {\tt insert}, and {\tt delete} to change the state of the database? How will
you test your program to ensure that it is correct?

\section*{Summary of the Required Deliverables}

You and your partner should always use a Git repository, hosted by Bitbucket, to store the schema visualizations, source
code, database files, and all of the other deliverables required by this assignment. The repository must be shared with
the instructor and the version control log should accurately reflect each student's contribution to this assignment. In
addition, this assignment invites your partnership to submit one printed version of the following deliverables; each
member should write and submit their own version of the first deliverable. Please see the instructor if you have
questions about the deliverables that you must submit for this assignment.

% \vspace*{-.05in}

\begin{enumerate}
    \setlength{\itemsep}{0pt}
  \item A two paragraph commentary on the work that each team member completed. 
  \item A commentary on the features, strengths, and weaknesses of the HyperSQL database engine.
  \item A technical diagram explaining the structure and behavior of the {\tt FindFile} program.
  \item A pseudo-ER diagram, produced by {\tt sqlt-graph}, of {\tt FindFile}'s relational schema.
  \item A description of the steps taken to compile and run the {\tt FindFile} program.
  \item The output from three separate runs of both phases of the {\tt FindFile} program.
  \item A natural-language description of the features provided by your chosen application.
  \item An explanation of how you use the ``dynamic SQL'' method to interact with HSQLDB.
  \item A pseudo-ER diagram, produced by {\tt sqlt-graph}, of your application's relational schema.
  \item The output from three separate runs of your database-centric application.
  \item A description of the steps that you took to ensure the correctness of your program.
  \item A reflection on the challenges that you faced when completing this laboratory assignment.
\end{enumerate}

% \vspace*{-.1in} 

In adherence to the Honor Code, students should complete this assignment while exclusively collaborating with the
other member of their team. While it is appropriate for students in this class---who are not in the same team---to have
high-level conversations about the assignment, it is necessary to distinguish carefully between the team that discusses
the principles underlying a problem with another team and the team that produces an assignment that is identical to, or
merely a variation on, the work of another team.  Deliverables from one team that are nearly identical to the work of
another team will be taken as evidence of violating Allegheny College's \mbox{Honor Code}.

% Before you turn in this assignment, you also must ensure that the course instructor has read access to your Bitbucket
% repository that is named according to the convention {\tt cs380F2014-<your user name>}.  Please see the instructor if
% you have any questions about this assignment. 

\end{document}
