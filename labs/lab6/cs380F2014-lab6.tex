%!TEX root=cs380F2014-lab6.tex
% mainfile: cs380F2014-lab6.tex

\input{labspre.tex}

\usepackage[compact]{titlesec}

\begin{document}
\MYTITLE{Laboratory Assignment Six \\ Implementing Database-Centric
Applications}
\MYHEADERS{Laboratory Assignment Six}{Due: October 29, 2014}

\section*{Introduction}

Chapter 5 of your textbook introduces advanced concepts in the structure query language (SQL), with a focus on topics
like accessing a database from a programming language. In this laboratory assignment, you will expand on the example in
Figure 5.1 of the textbook to ultimately implement your own database-centric application based on a schema that you
visualized and manipulated in the past two assignments.

\vspace*{-.05in}
\section*{Learning About the HyperSQL Database Engine}


\vspace*{-.05in}
\section*{Compiling and Using the FindFile Program}



\vspace*{-.05in}
\section*{Implementing a Database Centric Application}



\section*{Summary of the Required Deliverables}

You and your partner should always use a Git repository, hosted by Bitbucket, to store the schema visualizations and
all of the other deliverables required by this assignment. The repository must be shared with the instructor and
the version control log should accurately reflect each student's contribution to this assignment. In addition, this
assignment invites your partnership to submit one printed version of the following deliverables; each member should
write and submit their own version of the first deliverable. Please see the instructor if you have questions about
these matters.

% \vspace*{-.05in}

\begin{enumerate}
  \setlength{\itemsep}{0pt}
  \item A two paragraph commentary on the work that each team member completed. 
  \item An overview of how the {\tt sqlt-graph} command-line options influence the resulting diagrams.
  \item A comparison of the University visualizations from the textbook and the {\tt sqlt-graph} tool.
  \item A brief commentary on the visualizations produced for each of the provided schemas.
  \item The complete and commented SQL source code for the student-designed relational schema.
  \item The final visualization of the student-designed relational schema, as produced by {\tt sqlt-graph}.
  \item A reflection on the challenges that you faced when completing this laboratory assignment.
\end{enumerate}

% \vspace*{-.1in} 

In adherence to the Honor Code, students should complete this assignment while exclusively collaborating with the
other member of their team. While it is appropriate for students in this class---who are not in the same team---to have
high-level conversations about the assignment, it is necessary to distinguish carefully between the team that discusses
the principles underlying a problem with another team and the team that produces an assignment that is identical to, or
merely a variation on, the work of another team.  Deliverables from one team that are nearly identical to the work of
another team will be taken as evidence of violating Allegheny College's \mbox{Honor Code}.

% Before you turn in this assignment, you also must ensure that the course instructor has read access to your Bitbucket
% repository that is named according to the convention {\tt cs380F2014-<your user name>}.  Please see the instructor if
% you have any questions about this assignment. 

\end{document}
