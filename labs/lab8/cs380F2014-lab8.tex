%!TEX root=cs380F2014-lab8.tex
% mainfile: cs380F2014-lab8.tex

\input{labspre.tex}

\usepackage[compact]{titlesec}

\begin{document}
\MYTITLE{Laboratory Assignment Eight \\ Parsing XML Files Using the Document Object Model}
\MYHEADERS{Laboratory Assignment Eight}{Due: November 19, 2014}

\section*{Introduction}

Sections 23.4 and 23.5 of your textbook explain the steps that a software developer and/or a data analyst must take to
query, transform, and parse eXtensible markup language (XML) documents. In this laboratory assignment, you will download
and use two programs that leverage the tree-based document object model (DOM) to parse an XML document.  In addition,
you will extend the second program so that it can produce output in a more configurable fashion.  Finally, you and your
partner will conduct a simple experiment to measure the performance of DOM-based parsing and write a short report that
describes and explains the empirical trends that you identified.

\vspace*{-.05in}
\section*{Learning More About XML Parsing Methods}

In addition to review the aforementioned sections of your textbook---and any of relevant content in Chapter 23---you
should study the following Web site: \url{http://www.mkyong.com/java/how-to-read-xml-file-in-java-dom-parser/}. We will
use the simple example provided in this tutorial to start learning about how to parse an XML document.  You should also
carefully review Oracle's tutorial, called ``Reading XML Data into a DOM'', which is available from
\url{https://docs.oracle.com/javase/tutorial/jaxp/dom/readingXML.html}. To gain a better understanding of the different
DOM-based solutions that are available and to ensure that you understand some fundamental trade-offs associated with
using our chosen DOM parser, you should also read \url{https://docs.oracle.com/javase/tutorial/jaxp/dom/when.html}.

\vspace*{-.05in}
\section*{Simple DOM-Based Parsing of XML}

You and your partner should create a Git repository, using Bitbucket, that you can use to complete this laboratory
assignment.  To start off this project, one of you should go into the ``share'' repository for this course and run the
``{\tt git pull}'' command. After investigating the directories for this assignment, you will find the ``{\tt
  lab7/src/}'' and ``{\tt lab7/lib/}'' directories. What files do these directories contain? How does this program work?
Please note that the provided version of the benchmark does not yet include the features necessary to evaluate the
performance of default and XML serialization---you and your partner will add this code as part of the laboratory
assignment.


\vspace*{-.05in}
\section*{Using and Extending a Comprehensive XML Parser}



\section*{Summary of the Required Deliverables}

You and your partner should always use a Git repository, hosted by Bitbucket, to store the source code, XML files,
program output, and all of the other deliverables required by this assignment. The repository must be shared with the
course instructor and the version control log should accurately reflect each student's contribution to the assignment.
In addition, this assignment invites your partnership to submit one printed version of the following deliverables; each
member should write and submit their own version of the first deliverable. Please see the instructor if you have
questions about the deliverables that you must turn in for this laboratory assignment.

\vspace*{-.05in}
\begin{enumerate}
  \setlength{\itemsep}{0pt}
  \item A two paragraph commentary on the work that each team member completed. 
  \item A description of DOM-based XML parsing, examining its features, strengths, and weaknesses.
  \item 
  \item 
  \item Output from at least five runs of 
  \item A detailed report addressing the performance trends associated with DOM-based XML parsing.
  \item A reflection on the challenges that you faced when completing this laboratory assignment.
\end{enumerate}
\vspace*{-.05in}

In adherence to the Honor Code, students should complete this assignment while exclusively collaborating with the
other member of their team. While it is appropriate for students in this class---who are not in the same team---to have
high-level conversations about the assignment, it is necessary to distinguish carefully between the team that discusses
the principles underlying a problem with another team and the team that produces an assignment that is identical to, or
merely a variation on, the work of another team.  Deliverables from one team that are nearly identical to the work of
another team will be taken as evidence of violating Allegheny College's \mbox{Honor Code}.

% Before you turn in this assignment, you also must ensure that the course instructor has read access to your Bitbucket
% repository that is named according to the convention {\tt cs380F2014-<your user name>}.  Please see the instructor if
% you have any questions about this assignment. 

\end{document}
