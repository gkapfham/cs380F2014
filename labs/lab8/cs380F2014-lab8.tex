%!TEX root=cs380F2014-lab8.tex
% mainfile: cs380F2014-lab8.tex

%!TEX root=cs380F2014-lab1.tex
% mainfile: cs380F2014-lab1.tex 

% Typical usage (all UPPERCASE items are optional):
%       \input 580pre
%       \begin{document}
%       \MYTITLE{Title of document, e.g., Lab 1\\Due ...}
%       \MYHEADERS{short title}{other running head, e.g., due date}
%       \PURPOSE{Description of purpose}
%       \SUMMARY{Very short overview of assignment}
%       \DETAILS{Detailed description}
%         \SUBHEAD{if needed} ...
%         \SUBHEAD{if needed} ...
%          ...
%       \HANDIN{What to hand in and how}
%       \begin{checklist}
%       \item ...
%       \end{checklist}
% There is no need to include a "\documentstyle."
% However, there should be an "\end{document}."
%
%===========================================================
\documentclass[11pt,twoside,titlepage]{article}
%%NEED TO ADD epsf!!
\usepackage{threeparttop}
\usepackage{graphicx}
\usepackage{latexsym}
\usepackage{color}
\usepackage{listings}
\usepackage{fancyvrb}
%\usepackage{pgf,pgfarrows,pgfnodes,pgfautomata,pgfheaps,pgfshade}
\usepackage{tikz}
\usepackage[normalem]{ulem}
\tikzset{
    %Define standard arrow tip
%    >=stealth',
    %Define style for boxes
    oval/.style={
           rectangle,
           rounded corners,
           draw=black, very thick,
           text width=6.5em,
           minimum height=2em,
           text centered},
    % Define arrow style
    arr/.style={
           ->,
           thick,
           shorten <=2pt,
           shorten >=2pt,}
}
\usepackage[noend]{algorithmic}
\usepackage[noend]{algorithm}
\newcommand{\bfor}{{\bf for\ }}
\newcommand{\bthen}{{\bf then\ }}
\newcommand{\bwhile}{{\bf while\ }}
\newcommand{\btrue}{{\bf true\ }}
\newcommand{\bfalse}{{\bf false\ }}
\newcommand{\bto}{{\bf to\ }}
\newcommand{\bdo}{{\bf do\ }}
\newcommand{\bif}{{\bf if\ }}
\newcommand{\belse}{{\bf else\ }}
\newcommand{\band}{{\bf and\ }}
\newcommand{\breturn}{{\bf return\ }}
\newcommand{\mod}{{\rm mod}}
\renewcommand{\algorithmiccomment}[1]{$\rhd$ #1}
\newenvironment{checklist}{\par\noindent\hspace{-.25in}{\bf Checklist:}\renewcommand{\labelitemi}{$\Box$}%
\begin{itemize}}{\end{itemize}}
\pagestyle{threepartheadings}
\usepackage{url}
\usepackage{wrapfig}
% removing the standard hyperref to avoid the horrible boxes
%\usepackage{hyperref}
\usepackage[hidelinks]{hyperref}
% added in the dtklogos for the bibtex formatting
\usepackage{dtklogos}
%=========================
% One-inch margins everywhere
%=========================
\setlength{\topmargin}{0in}
\setlength{\textheight}{8.5in}
\setlength{\oddsidemargin}{0in}
\setlength{\evensidemargin}{0in}
\setlength{\textwidth}{6.5in}
%===============================
%===============================
% Macro for document title:
%===============================
\newcommand{\MYTITLE}[1]%
   {\begin{center}
     \begin{center}
     \bf
     CMPSC 380\\Principles of Database Systems\\
     Fall 2014
     \medskip
     \end{center}
     \bf
     #1
     \end{center}
}
%================================
% Macro for headings:
%================================
\newcommand{\MYHEADERS}[2]%
   {\lhead{#1}
    \rhead{#2}
    %\immediate\write16{}
    %\immediate\write16{DATE OF HANDOUT?}
    %\read16 to \dateofhandout
    \def \dateofhandout {September 3, 2014}
    \lfoot{\sc Handed out on \dateofhandout}
    %\immediate\write16{}
    %\immediate\write16{HANDOUT NUMBER?}
    %\read16 to\handoutnum
    \def \handoutnum {3}
    \rfoot{Handout \handoutnum}
   }

%================================
% Macro for bold italic:
%================================
\newcommand{\bit}[1]{{\textit{\textbf{#1}}}}

%=========================
% Non-zero paragraph skips.
%=========================
\setlength{\parskip}{1ex}

%=========================
% Create various environments:
%=========================
\newcommand{\PURPOSE}{\par\noindent\hspace{-.25in}{\bf Purpose:\ }}
\newcommand{\SUMMARY}{\par\noindent\hspace{-.25in}{\bf Summary:\ }}
\newcommand{\DETAILS}{\par\noindent\hspace{-.25in}{\bf Details:\ }}
\newcommand{\HANDIN}{\par\noindent\hspace{-.25in}{\bf Hand in:\ }}
\newcommand{\SUBHEAD}[1]{\bigskip\par\noindent\hspace{-.1in}{\sc #1}\\}
%\newenvironment{CHECKLIST}{\begin{itemize}}{\end{itemize}}


\usepackage[compact]{titlesec}

\begin{document}
\MYTITLE{Laboratory Assignment Eight \\ Parsing XML Files Using the Document Object Model}
\MYHEADERS{Laboratory Assignment Eight}{Due: November 19, 2014}

\section*{Introduction}

Sections 23.4 and 23.5 of your textbook explain the steps that a software developer and/or a data analyst must take to
query, transform, and parse eXtensible markup language (XML) documents. In this laboratory assignment, you will download
and use two programs that leverage the tree-based document object model (DOM) to parse an XML document.  In addition,
you will extend the second program so that it can produce output in a more configurable fashion.  Finally, you and your
partner will conduct a simple experiment to measure the performance of DOM-based parsing and write a short report that
describes and explains the empirical trends that you identified.

\vspace*{-.05in}
\section*{Learning More About XML Parsing Methods}

In addition to review the aforementioned sections of your textbook---and any of relevant content in Chapter 23---you
should study the following Web site: \url{http://www.mkyong.com/java/how-to-read-xml-file-in-java-dom-parser/}. We will
use the simple example provided in this tutorial to start learning about how to parse an XML document.  You should also
carefully review Oracle's tutorial, called ``Reading XML Data into a DOM'', which is available from
\url{https://docs.oracle.com/javase/tutorial/jaxp/dom/readingXML.html}. To gain a better understanding of the different
DOM-based solutions that are available and to ensure that you understand some fundamental trade-offs associated with
using our chosen DOM parser, you should also read \url{https://docs.oracle.com/javase/tutorial/jaxp/dom/when.html}.

\vspace*{-.05in}
\section*{Simple DOM-Based Parsing of XML}

You and your partner should create a Git repository, using Bitbucket, that you can use to complete this laboratory
assignment.  To start off this project, one of you should go into the ``share'' repository for this course and run the
``{\tt git pull}'' command. After investigating the directories for this assignment, you will find the ``{\tt
  lab8/}'' directory. What files does this directory contain?  Please see the course instructor if your repository does
not have two XML files and two Java programs.

You are your partner should used a text editor, like GVim, to view the file called ``{\tt staff.xml}''.  What are the
contents of this file?  Next, you should use GVim to customize the contents of this file so that they correspond to
staff members at Allegheny College---you can take a guess for any attributes of a staff member that you do not know.  In
addition, you are welcome to add in additional data and meta-data as long as it adheres to the standard set forth in the
original version of ``{\tt staff.xml}''. Now, you should compile and run the program called ``{\tt
  ReadXMLFileWithDOM.java}'' How does this program work? What are the limitations associated with this program's
implementation? What is the output of this program? Does the program seem to work correctly?

\vspace*{-.05in}
\section*{Using and Extending a Comprehensive XML Parser}



\section*{Summary of the Required Deliverables}

You and your partner should always use a Git repository, hosted by Bitbucket, to store the source code, XML files,
program output, and all of the other deliverables required by this assignment. The repository must be shared with the
course instructor and the version control log should accurately reflect each student's contribution to the assignment.
In addition, this assignment invites your partnership to submit one printed version of the following deliverables; each
member should write and submit their own version of the first deliverable. Please see the instructor if you have
questions about the deliverables that you must turn in for this laboratory assignment.

\vspace*{-.05in}
\begin{enumerate}
  \setlength{\itemsep}{0pt}
  \item A two paragraph commentary on the work that each team member completed. 
  \item A description of DOM-based XML parsing, examining its features, strengths, and weaknesses.
  \item A discussion of how normalization can influence the output of a DOM-based XML parser.
  \item The final version of the {\tt staff.xml} XML file that you customized and extended as appropriate.
  \item The output from running the {\tt ReadXMLFileWithDOM} program in the terminal window.
  \item A report addressing the performance trends associated with DOM-based XML parsing.
  \item A reflection on the challenges that you faced when completing this laboratory assignment.
\end{enumerate}
\vspace*{-.05in}

In adherence to the Honor Code, students should complete this assignment while exclusively collaborating with the
other member of their team. While it is appropriate for students in this class---who are not in the same team---to have
high-level conversations about the assignment, it is necessary to distinguish carefully between the team that discusses
the principles underlying a problem with another team and the team that produces an assignment that is identical to, or
merely a variation on, the work of another team.  Deliverables from one team that are nearly identical to the work of
another team will be taken as evidence of violating Allegheny College's \mbox{Honor Code}.

% Before you turn in this assignment, you also must ensure that the course instructor has read access to your Bitbucket
% repository that is named according to the convention {\tt cs380F2014-<your user name>}.  Please see the instructor if
% you have any questions about this assignment. 

\end{document}
